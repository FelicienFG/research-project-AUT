
\section{Related Works}
\label{sec:literature}

\cite{ala} is another good paper.

\subsection{Systematic Literature Review process}

The SLR process~\cite{kitchenham2009systematic}.

\paraheading{\textbf{Scoping}}
The following research questions were used to conduct this SLR. 
\begin{itemize}
    \item RQ1
    \item RQ2
\end{itemize}
Extant literature does not sufficiently answer this question (provide a summary of current secondary studies).

\paraheading{\textbf{Planning} involved breaking down the RQs into individual search terms.} We identified the following search terms (concepts) which were combined in the following way. Some practice search runs were used to ensure that the search terms were correct and yielded correct results.
Preliminary inclusion and exclusion criteria were ....

\paraheading{It was decided that search results would be recorded by the use of ... (provide details)}.

\paraheading{\textbf{Searching}}

\paraheading{Searching was conducted on the following databases, and XYZ number of results were returned.} 

\paraheading{\textbf{Screening and Eligibility}}
Prune the list to identify the final list (max 20 papers). 

\subsection{Findings of the Literature Review}

\paraheading{The works reviewed were compared on the following metrics.} 
\begin{itemize}
    \item \textbf{Metric 1}: justification
    \item \textbf{Metric 2}: justification
    \item $\ldots$ (add as many as needed)
\end{itemize}

A comparison of the works was carried out and the overall results are illustrated in Table~\ref{tab:slrtable}.

\paraheading{Provide a summary of what the results say.}

\paraheading{Show how your work is novel.}



\begin{table}[htb]
    \centering
    \begin{tabular}[]{|l|p{0.20\linewidth}|p{0.35\linewidth}|}
        \hline
        $\mathbf{References}$ & $\mathbf{Category}$ & $\mathbf{Method}$ \\
        \hline
        \cite{CaoStretchingDAGs2020}\cite{GuanDAGfluid2021} & Decomposition & task segmentation and processor
        exclusivity for critical path\cite{CaoStretchingDAGs2020}, and fluid scheduling\cite{GuanDAGfluid2021}\\
        \hline
        \cite{WangGEDFDag2019}\cite{SchmidResponseDAGThreadpools2021}\cite{zhao2022dag}\cite{ChenDAGorder2023} & Partitioned/ Federated & Federated scheduling for inter-DAG scheduling and GEDF for intra-DAG\cite{WangGEDFDag2019}, 
        global preemptive fixed-priority scheduling using assigned thread-workers based on DAG workload\cite{SchmidResponseDAGThreadpools2021},
        workload and no inter-task interference based processor assignment with CPC model for intra-task priority assignment\cite{zhao2022dag},
        federated and order-based intra-task priority assignment based on workload and communication delays\cite{ChenDAGorder2023}\\
        \hline
        \cite{he2019intra} & Global & G-RM and G-EDF for inter-task and priority assignment by maximizing intra-task parralelism for intra-task scheduling\\
        \hline
        \cite{ChangMinWRCTBoundILP2022} & ILP & Only interested in intra-task interference, uses Integer Linear Programming to minimize the makespan\\
        \hline
        \cite{yano2021work}\cite{lee2021DAGDeeplearning} & Reinforcement Learning & Q-learning with partitioning at the intra-task level using EST heuristic\cite{yano2021work},
        GCN\footnotemark and sequential encoding priority assignment for a single DAG task and then use a work-conserving GFPS\footnotemark to assign tasks to processors\cite{lee2021DAGDeeplearning}\\
        \hline
        \cite{Igarashi2020HeuristicContFree}\cite{Yano2021ContentionFree} & mixed global / partitioned & blocks access from main tasks to certain cores to avoid contention
        by interpreting I/O operations as tasks with precedence constraints and parallelizing the main tasks\\
        \hline
    \end{tabular}
    \caption{Summary table for DAG task scheduling.}
    \label{tab:sum_table}
\end{table}


%\begin{table}[htb]
%\caption{Systematic Literature Review Results (Generate tables from \url{www.tablesgenerator.com})}
%\label{tab:slrtable}
%\begin{tabular}{|l|l|l|l|l|}
%\hline
%             & Metric 1 & Metric 2 & Metric 3 & Metric 4 \\ \hline
%{[}Work 1{]} &          &          &          &          \\ \hline
%{[}Work 2{]} &          &          &          &          \\ \hline
%{[}Work 3{]} &          &          &          &          \\ \hline
%{[}Work 4{]} &          &          &          &          \\ \hline
%\end{tabular}
%\end{table}
