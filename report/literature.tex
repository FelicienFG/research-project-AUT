
\section{Related Works}
\label{sec:literature}

\subsection{Systematic Literature Review process}

\paraheading{\textbf{Scoping}}

This SLR aims at tackling RQ1. More precisely, the following research questions will be answered:

\begin{itemize}
    \item [RQ1.1] What is the current state-of-the-Art for DAG task scheduling with precedence constraints ?
    \item [RQ1.2] How has LET been used in scheduling event-chains ?
    \item [RQ1.3] What machine learning techniques have been used for scheduling tasks on real-time systems ?
\end{itemize}
It will also be shown how the literature doesn't provide 
a complete answer to RQ2, hence the contributions of this paper.\\

From these research questions, several concepts have been isolated,
namely, time-triggered tasks, the nature of the system (real-time multicore system),
the scheduling of tasks, DAG tasks, and machine learning.
The recording of the search results were done using the BibTeX LateX plugin
combine with the google scholar "cite" feature.

Searching was conducted using the IEEE and ACM databases.
According to the concepts identified above, 
the keyword chain used for searching was 
"("real-time" OR "real time") AND 
"system" AND ("time-triggered" OR "time triggered" OR "DAG" OR "Directed Acyclic Graph" OR "LET" OR "Logical Execution Time" OR "event chain" OR "event-chain") 
AND "task" 
AND ("scheduling" OR "scheduler" OR "schedule") 
AND ("multi-processors" OR "multi-cores" OR "multi processors" OR 
"multi cores" OR "multi-processor" OR "multi processor" OR 
"multi-core" OR "multi core")".

The search produced 3,549 results on the IEEE database

exclusion : past 5 years : IEEE --> down to 1,171 
            heterogeneous not in title + abstract : IEEE --> down to 999
            mixed critical* not in title + abstract : IEEE --> down to 952
            scheduling or scheduler or schedule in title but not "energy" :  IEEE --> down to 155 and 149 when just considering conference and journal papers (not early access)
            
            removing those not about real-time system,
            not about proposing a scheduling algorithm,
            not about DAG nor LET tasks or event-chains : IEEE --> down to 21
            After reading the complete articles --> 19.

\paragraph{}

\subsection{Findings of the Literature Review}

\paraheading{The works reviewed were compared on the following metrics.} 
\begin{itemize}
    \item \textbf{Utilization Bound}: useful to see which algorithm is more efficient at using the available resources.
    \item \textbf{Acceptance Ratio}: it shows how optimal (see Section \ref{sec:bg}) a scheduling algorithm is.
    \item \textbf{Makespan}: for DAG task scheduling, widely used in the literature.
    \item \textbf{Runtime Overhead}: some scheduling algorithms can show promising results theoretically but are practically very slow because 
    of their complexity adding runtime overhead on the scheduler, this metric will not be a number but rather an amount such as minimal, practical, non-practical.
\end{itemize}
Every metric used here have also been chosen for their prevalence in the literature.

A comparison of the works was carried out and the overall results are illustrated in Table~\ref{tab:slrtable}.
 
In \cite{guan2021DAGfluid}, authors 
use fluid scheduling to schedule multiple DAG tasks
on a multicore system. 
Fluid scheduling has been used in previous work for independent
time-triggered task scheduling\cite{baruah1993PFair}\cite{cho2006LLREF} 
but very few consider DAG tasks. Fluid-scheduling is known for producing optimal scheduling algorithms.
Their method decomposes a DAG task into several sequential segments
in which the subtasks will execute according to the fluid scheduling 
model. Although their algorithm significantly outperforms
existing algorithms, the main limitation that is common to all fluid-based
scheduling algorithm is the runtime overhead induced by the fluid-scheduling model.
Although authors in \cite{guan2021DAGfluid} briefly explain how 
to transform their scheduling algorithm to a non-fluid one for practical implementation,
they do not evaluate the overhead caused by the frequent task migrations
and preemptions. Also, their algorithm only considers DAG task with implicit deadlines
(D = T) which makes the response-time analysis simpler but to the cost
of generalizability.

As a follow up, authors in \cite{GuanFluidDag2022} 
extend the fluid scheduling algorithm in \cite{guan2021DAGfluid}
to constrained and arbitrary deadline tasks,
especially focusing on DAG tasks with a deadline greater than their period.
Their main contributions are their new scheduling algorithm that 
performs better than existing methods in terms of acceptance ratio,
and producing the first theoretical capacity bound for DAG tasks
with deadlines greater than their periods.
However, the authors still don't provide any evaluation 
on the amount of runtime overhead their scheduling algorithm implementation
produces which generally lowers the actual acceptance ratio
of the algorithm. 

Instead of considering fluid-scheduling,
a popular scheduling method is federated scheduling.
Federated scheduling is based on the idea
of assigning heavy tasks ($U > 1$) to multiple cores
for the whole duration of the tasks' executions,
and assigning light tasks ($U \le 1$) to execute on
cores that have not been assigned a heavy task.
Although it is popular, it suffers from a resource wasting problem,
especially when the difference between the critical path's length 
and the deadline is small,
which many papers aim at solving
\cite{Guan2023FederatedNew}
\cite{jiangUtilTensityBound}
\cite{JiangVirtuallyFederatedSched2021}
\cite{Jiang2023SchedVirtualProcs}
\cite{Kobayashi2023FedBundledDagsched}
\cite{Shi2024DagExecGroups}
\cite{He2023DegreeOfParallelism}.

\cite{jiangUtilTensityBound}, for instance, 
consider federated scheduling and GEDF
and introduces a better metric called the util-tensity bound
that extends the concept of capacity bound
to have a better schedulability test.
Based on this newly derived bound, the authors 
propose an extension to the classic federated algorithm,
with very low tensity tasks being scheduled with GEDF, 
tasks with high-utilization and relatively high tensities are scheduled
using the classic federated scheduling and low utilization
tasks with relatively high tensities are scheduled using partitioned-EDF.
Their algorithm, based on their newly derived bound, effectively improves
the system schedulability of DAG tasks and reduces the resource wasting 
problem of federated scheduling. The main limitation
of this paper is that they only consider GEDF 
for their util-tensity bound and also only consider implicit deadline DAG tasks.

This problem of resource wasting in federated scheduling
is also tackled in \cite{Kobayashi2023FedBundledDagsched}
where the authors propose a federated and bundled-based scheduling
algorithm which enhances the schedulability of DAG tasks
compared to existing federated scheduling algorithms.
Their method consists of using federated scheduling for
tasks with high critical path to deadline ratio and bundled
scheduling for tasks with low critical path to deadline ratio.
Unfortunately, this paper only looks at 3 DAG tasks to evaluate
their algorithm which is a really small amount and is not 
representative of the different DAG tasks that can exist.

Authors in \cite{JiangVirtuallyFederatedSched2021}
take another approach by proposingg a virtually-federated 
scheduling algorithm that leverages the advantages
of federated scheduling while improving the acceptance 
ratio for DAG tasks, outperforming existing algorithms.


\begin{table}
    \centering
    \begin{tabular}[]{|l|p{0.20\linewidth}|p{0.20\linewidth}|p{0.20\linewidth}|}
        \hline
        \textbf{Reference} & \textbf{Scheduling technique} & \textbf{Task type} & \textbf{Scope (intra/inter/both)}\\
        \hline
        \cite{guan2021DAGfluid} & fluid & implicit deadline & inter\\
        \hline
        \cite{He2019DagIntra} & priority-list & constrained deadline & intra \\
        \hline
        \cite{Kobayashi2023FedBundledDagsched} & federated and bundled-based & constrained deadline & inter\\
        \hline
        \cite{Xiao2019}  & clustering & constrained deadline & intra\\ 
        \hline
        \cite{Igarashi2020HeuristicContentionFree}  & priority-list & LET constrained deadline & both \\
        \hline
        \cite{jiangUtilTensityBound}  & federated and GEDF and PEDF & implicit deadline & inter\\
        \hline
        \cite{JiangDecompoSchedParallelTask} & Decomposition-based & implicit deadline & inter \\
        \hline
        \cite{He2023DegreeOfParallelism} & federated-based & constrained deadlines & inter \\
        \hline
        \cite{Shi2024DagExecGroups}  & partitioned / clustering & constrained deadlines & intra\\
        \hline
        \cite{Guan2023FederatedNew}  & federated & arbitrary deadline & inter\\
        \hline
        \cite{Zhao2024GATDRLmodel} & DRL & constrained deadline & intra\\
        \hline
        \cite{Xu2023DRLtaskSched} & DRL & non-DAG implicit deadline & inter\\
        \hline
        \cite{Zhao2022DAGsched} & priority-list and federated & constrained deadline & both \\
        \hline
        \cite{Lee2021GlobalDagSchedDRL} & DRL & constrained deadline & intra\\
        \hline
        \cite{Jiang2023SchedVirtualProcs} & federated-based & constrained deadline & inter\\
        \hline
        \cite{GuanFluidDag2022} & fluid & constrained/arbitrary deadline & inter\\
        \hline
        \cite{GuanFRTDS2020RL} & DRL & constrained deadline & intra \\
        \hline
        \cite{JiangVirtuallyFederatedSched2021} & federated-based & constrained deadline & inter\\
        \hline
        \cite{Pazzaglia2021DMALETtransfer} & Mixed ILP & LET, constrained deadline & inter\\
        \hline
        \textbf{Total: 19} & \textbf{DRL: 4, Federated: 7, Fluid: 2, ILP: 1, Priority-List(intra): 3, Clustering: 2, Decomposition: 1}
        & \textbf{implicit: 4, constrained: 13, arbitrary: 2} & \textbf{inter: 11, intra: 6, both: 2} \\
        \hline
    \end{tabular}
    \caption{SLR summary table}
    \label{tab:slt_sum_table}
\end{table}



