\section{Limitations}
~
\\


Although the DAG generator used is indeed used in the literature\cite{Lee2021GlobalDagSchedDRL}\cite{zhao2020DAGsched}\cite{Zhao2022DAGsched},
the generator is not made for generating a number of DAG tasks
with a fixed number of nodes. Hence, 
using the generator as it has been used might not
lead to a dataset as diverse as what is done by \citet{Zhao2024GATDRLmodel} for instance,
and it might not be as qualitative as the authors of the generator claim it to be.
Also, no evaluation of the model using multiple values of the parallelism
parameter $p$ has been done, which could have been interesting 
to look at to see the impact of this parameter, and of how parallelizable 
a DAG task is, on the performance.
The amount of DAG tasks used as a test set is 400 which 
is lower than what the other papers used, 600 in \citet{Zhao2024GATDRLmodel}
and 1000 in \citet{Lee2021GlobalDagSchedDRL}, which makes randomness
and dataset bias have a bigger influence than in those papers.

For the evaluation, 
it has only been done using a train set and test set,
no cross-validation such as K-fold has been done which could have
enhanced the interpretation of the results, reducing the impact
of dataset bias in the performance results.

Furthermore, 
the supervised learning method might still be an option
for this kind of problem, as the main problem isn't the supervised
learning method per se, but rather the computation of the output labels. 
Also, although the ILP method computes
optimal schedules, it does take a long time to do so and it is not scalable,
making the model training not scalable which can
become problematic if we want to train the model
on DAGs with 100, 500 or 1000 nodes.
