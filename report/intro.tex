\section{Introduction}
\label{sec:intro}

Real-time systems are utilized in various domains such as air traffic 
control, public transportation, and automated vehicles. Unlike non-real-time systems, 
tasks in real-time systems must be both functionally correct and meet strict 
 execution time constraints, known as deadlines. Failure 
to meet these deadlines can lead to severe consequences. The critical 
nature of these systems necessitates designing the system 
architecture with a focus on time and incorporating fault tolerance 
to ensure high reliability.

One example of such architecture is the time-triggered 
architecture (TTA)\cite{kopetz2003tta}\cite{kopetz1998timetriggered}, which offers a fault-tolerant 
communication protocol and a precise timing system to synchronize different electronic control units. 
Increasingly, these real-time system architectures are enhancing their 
computational resources by transitioning to multiprocessor systems. 
This shift from uniprocessor to multiprocessor systems addresses 
the growing complexity and computational demands of tasks executed 
on these systems (e.g., autonomous cars, computer vision systems), aiming to reduce both the execution time of these 
tasks and the required resources to run them\cite{maiza2019survey}.

Hence, an increasing number of real-time systems are 
utilizing multi-core hardware to parallelize their tasks 
and convert sequential programs into parallelized ones using 
frameworks such as OpenMP
\footnote{OpenMP (2011) OpenMP Application Program Interface v3.1. 
\url{http://www.openmp.org/mp-documents/OpenMP3.1.pdf}} to do so. 
Unfortunately, in most real-life scenarios, the number of available 
processors/cores is fewer than the number of tasks/subtasks that 
can be executed in parallel (i.e., independent tasks). This means 
that not all independent tasks can be executed simultaneously on 
the system, raising the question: which task should be executed first?

This question is particularly important in a real-time context
because having the wrong execution order, or schedule, could lead 
to, at best, a slow system, and at worst, deadline misses, which 
can have fatal repercussions. In the case of a self-driving car 
system, for instance, a slight delay of 500 ms in detecting a pedestrian 
crossing the road can, in some cases, be enough to drive over 
the pedestrian or cause a car accident. Note that the resources of 
real-time systems are scarce and limited, which is why using as 
little processing power as possible while ensuring that tasks meet 
their deadlines is of crucial importance.

The extreme case of this scheduling problem arises when only one 
processor is available to execute tasks. This is known as task 
scheduling on a uniprocessor, and \cite{liu1973scheduling} 
provided two major priority policies: Rate Monotonic (RM) and 
Earliest Deadline First (EDF) for scheduling independent periodic tasks. 
However, when considering multiple processors, the scheduling 
problem becomes much more complex, and different task models must 
be considered.

A prevalent task model when considering periodic tasks that can be parallelized is the 
Directed Acyclic Graph (DAG) task model which arises when a time-triggered task
can be parallelized into subtasks which are the nodes of the graph,
thus reducing the worst-case execution time of the reccurent task when executed on a designated set of cores.
Those nodes have dependency constraints which are modeled by the directed edges between the nodes.
The DAG task model is used to model tasks that are parallelizable\cite{baruah2012DAGdef},
fitting the ever-increasing multicore architectures found in today's real-time system architectures. 

Given that the problem of scheduling independent tasks or dependency-constrained groups of jobs (i.e., DAGs) is NP-hard\footnote{If a problem is 
NP-hard, it means that it is impossible to find a solution in 
polynomial time complexity, i.e., solutions are not scalable}\cite{du1989schedNPhard}
\cite{ULLMAN1975NPhard}, 
people have resorted to either heuristics 
to partially solve the problem,
or the optimal but not scalable Integer Linear Programming
(ILP) method.

Consequently, machine learning will be considered here as it can 
better approximate the unattainable perfect solution while being 
scalable in terms of computing time after the training phase\cite{Zhao2024GATDRLmodel}\cite{Lee2021GlobalDagSchedDRL}. 
More specifically, the research questions are the following:

\begin{itemize}
    \item [RQ1] What is the current state-of-the-Art for DAG tasks scheduling ?        
    \item [RQ2] What machine learning  techniques are used for DAG task scheduling ?
    \item [RQ3]  Can machine learning be a better solution to schedule DAG tasks ?
            \begin{itemize}
                \item [RQ3.1] Can a machine learning solution compare to state-of-the art heuristics for scheduling Directed Acyclic Graph tasks ?
                \item [RQ3.2] Can a machine learning solution compare to an ILP solution while being more scalable ?
            \end{itemize}    
\end{itemize}

To achieve this, the background section will introduce various 
technical terms, concepts, and fundamental algorithms. 
Following this, a systematic literature review will be conducted to address RQ1 and RQ2, 
and finally, the artifact and experimental design, results, and conclusion will 
be presented to answer RQ3.



\paraheading{The solution we propose has the following features..}

\paraheading{The primary contributions of this paper are:}
