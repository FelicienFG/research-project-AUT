\section{Introduction}
\label{sec:intro}

Real-time systems are utilized in various domains such as air traffic 
control, public transportation, and automated vehicles. Unlike non-real-time systems, 
tasks in real-time systems must be both functionally correct and meet strict 
(or flexible) execution time constraints, known as deadlines. Failure 
to meet these deadlines can lead to severe consequences. The critical 
nature of these systems necessitates designing the system 
architecture with a focus on time and incorporating fault tolerance 
to ensure high reliability.

One example of such architecture is the time-triggered 
architecture (TTA)\cite{kopetz2003tta}\cite{kopetz1998timetriggered}, which offers a fault-tolerant communication protocol and a precise timing system to synchronize different electronic control units. Developing and running tasks on these architectures require in-depth knowledge of the system and its architecture, which complicates code reusability and scalability when adding hardware resources or upgrading to a larger system.

To address these issues, the Automotive Open System Architecture 
(AUTOSAR\footnote{\url{https://www.autosar.org/}}) was developed. 
AUTOSAR introduces layers of abstraction between hardware, firmware, 
and software, enhancing software reusability and hardware 
scalability across different systems while maintaining safety and 
security standards. It is now the most widely used architecture among car 
manufacturers, with notable core partners including BMW, Ford, and Toyota.

Scalability, in particular, plays a crucial role in modern 
real-time systems. Increasingly, real-time systems such as 
autonomous cars or computer vision systems are enhancing their 
computational resources by transitioning to multiprocessor systems. 
This shift from uniprocessor to multiprocessor systems addresses 
the growing complexity and computational demands of tasks executed 
on these systems, aiming to reduce both the execution time of these 
tasks and the required resources\cite{maiza2019survey}.

Hence, an increasing number of real-time systems are 
utilizing multi-core hardware to parallelize their tasks 
and convert sequential programs into parallelized ones using 
frameworks such as OpenMP
\footnote{OpenMP (2011) OpenMP Application Program Interface v3.1. 
\url{http://www.openmp.org/mp-documents/OpenMP3.1.pdf}}. 
Unfortunately, in most real-life scenarios, the number of available 
processors/cores is fewer than the number of tasks/subtasks that 
can be executed in parallel (i.e., independent tasks). This means 
that not all independent tasks can be executed simultaneously on 
the system, raising the question: which task should be executed first?

This question is particularly important in a real-time context 
because having the wrong execution order, or schedule, could lead 
to, at best, a slow system, and at worst, deadline misses, which 
can have fatal repercussions. In the case of a self-driving car 
system, for instance, a slight delay of 500 ms in detecting a pedestrian 
crossing the road can, in some cases, be enough to drive over 
the pedestrian or cause a car accident. Note that the resources of 
real-time systems are scarce and limited, which is why using as 
little processing power as possible while ensuring that tasks meet 
their deadlines is of crucial importance.

The extreme case of this scheduling problem arises when only one 
processor is available to execute tasks. This is known as task 
scheduling on a uniprocessor, and \cite{liu1973scheduling} 
provided two major priority policies: Rate Monotonic (RM) and 
Earliest Deadline First (EDF) for scheduling periodic tasks. 
However, when considering multiple processors, the scheduling 
problem becomes much more complex, and different task models must 
be considered.

A prevalent task model is the time-triggered task model, 
which specifies tasks that execute periodically and is well-suited 
for time-triggered systems. Another type of task is the Logical 
Execution Time (LET) task. The LET paradigm is based on the 
time-triggered paradigm and was originally introduced by the 
Giotto real-time programming language\cite{henzinger2003giotto} and 
later refined by \cite{henzinger2009distributed} into the 
Hierarchical Timing Language (HTL). The main principle behind the 
LET paradigm is that each task's inputs and outputs are read and 
written in zero time, i.e., constant time.

The benefits of using LET are twofold. Firstly, the zero-time 
communication semantics greatly improve the predictability of the 
overall system and make I/O operations (i.e., memory access) on 
shared resources deterministic, which is crucial in real-time 
systems due to the highly negative impact that memory access 
contentions can have on the system\cite{nagalakshmi2016impact}. 
Secondly, using LET in programming also provides a layer of 
abstraction that facilitates the direct translation from modeling 
to implementation, thus ensuring the implementation of timing 
requirements, enhancing code maintenance, and producing a less 
error-prone code base\cite{kirsch2012logical}.

One drawback of LET is its implementation overhead, which increases 
the execution times of tasks due to the zero-time communication 
semantics\cite{biondi2018LETruntimeoverhead}. Despite this drawback, 
the advantages of LET make it attractive for real-time systems\cite{gemlau2021systemLET}, 
which is why the focus here will be on time-triggered and LET 
task scheduling on multi-core systems. Given that the problem of 
scheduling independent tasks is NP-hard\footnote{If a problem is 
NP-hard, it means that it is very unlikely to find a solution in 
polynomial time complexity, i.e., solutions are not scalable}\cite{du1989schedNPhard}, 
no scalable optimal algorithm exists. Therefore, heuristics are 
used to partially solve the problem.

Consequently, machine learning will be considered here as it can 
better approximate the unattainable perfect solution while being 
scalable in terms of computing time after the training phase. In 
other words, the research questions are:

\begin{itemize}
    \item [RQ1] What is the current state-of-the-Art in scheduling event-chains of tasks ?
            \begin{itemize}
                \item [RQ1.1] What is the current state-of-the-Art for DAG task scheduling ?
                \item [RQ1.2] How has LET been used in scheduling event-chains ?
                \item [RQ1.3] What machine learning  techniques are used for DAG task scheduling ?
            \end{itemize}
    \item [RQ2]  Can machine learning be a better solution to schedule event-chains of tasks ?
            \begin{itemize}
                \item [RQ2.1] Can a machine learning solution compare to state-of-the art heuristics for scheduling Directed Acyclic Graph tasks ?
                \item [RQ2.2] Can a machine learning solution compare to ILP solutions while being more scalable ?
            \end{itemize}    
\end{itemize}

To achieve this, the background section will introduce various 
technical terms, concepts, and fundamental algorithms. 
Following this, a systematic literature review will be conducted to address R1, 
and finally, the artifact and experimental design, results, and conclusion will 
be presented to answer R2.



\paraheading{The solution we propose has the following features..}

\paraheading{The primary contributions of this paper are:}
