\documentclass{article}
\usepackage{biblatex}
\usepackage{longtable}

\bibliography{../references}

\begin{document}

\begin{longtable}{|l|p{0.25\linewidth}|p{0.25\linewidth}|p{0.20\linewidth}|p{0.25\linewidth}|}
        \hline
        \textbf{Reference} & \textbf{Motivation} & \textbf{Contribution(s)} & \textbf{Limitation(s)} & \textbf{Methodology Summary}\\
        \hline
        \cite{guan2021DAGfluid}  & DAG tasks scheduling is getting more popular and fluid scheduling performs great theoretically & Provide a DAG-fluid scheduling algorithm
        that performs way better in terms of acceptance ratio then previous algorithms & Fluid scheduling is unpractical and introduces a lot of overhead and task migrations, also only for implicit deadlines tasks
        & fluid-based algorithm where it decomposes a DAG's subtasks into multiple sequential segments\\
        \hline
        \cite{He2019DagIntra}  & DAGs are popular but no one looked at the intra-task execution order to leverage the graph structure & proposes a priority list scheduling algorithm for a single DAG task
        which performs better than SOTA in terms of makespan & no comparison with optimal priority assignment algorithms / optimal schedules.  & uses the length (in terms of wcets) of each paths passing through the current vertex to assign the priority to the current vertex,
        the higher the length, the higher priority \\
        \hline
        \cite{Kobayashi2023FedBundledDagsched} & Federated scheduling for DAG tasks is has proved efficient but 
        for tasks where the difference between the critical path and the deadline is small, it
        can lead to over-allocating cores.  & proposed a fedrated and bundled-based scheduling algorithm to avoid this problem and enhanced the schedulability of DAG tasks using their algorithms & They only compare their method with an example of a dag task set comprised of 3 dag tasks. & Uses federated scheduling for tasks with high critical path to deadline ratio and bundled scheduling for tasks with low critical path to deadline ratio. \\
        \hline
        \cite{Xiao2019} & DAG task scheduling is NP-hard so one can only approximate the optimal algorithm (when considering polynomial timed algorithms)
        and not a lot has been done on scheduling parrallel reccuring tasks & Introduces a scheduling algorithm 'MAS' that shortens the makespan of recurring DAG tasks compared to EDF & only compares EDF and MAS using one example of a DAG task for makespans and also only compares with EDF.
        Even though MAS shortens the makespan, it is less scalable than comparable algorithms. & The MAS algorithm combines techniques from clustering scheduling and task duplication algorithm and evaluates it on an actual simulation object (the TMS320C6678) so that the measurement are close to the real-life measurement you would get on the real system.\\
        \hline
        \cite{Igarashi2020HeuristicContentionFree}  & The use of multi-core systems can induce contentions because of shared memory / cache. This can lead to non-determinim and unpredictable behavior which
        violates the safety requirements of real-time systems, hence using LET tasks to fix those contentions 
        but LET also suffers from additional execution time being added because of the implementation overhead.
        & Proposes a DAG LET tasks scheduling algorithm based that avoids contentions while reducing the running time overhead
        due to LET implementation & They are using a multiple-clusters, multi-core architecture to evaluate their scheduling algorithm
        but only consider one cluster. & Uses a minimum-laxity-first priority assignment for intra-tasks and Earliest Finish time (EFT) for assigning tasks to cores.
        Considers multi-rate dags and reducing the LET interval to decrease the makespan. \\
        \hline
        \cite{jiangUtilTensityBound}  & The capacity-bound is a bound used as a performance measurement 
        but also as a schedulability test for DAG scheduling. However, 
        it uses the same bound to bound both the normalized utilization and the tensity (critical path over the deadline) of a DAG task which 
        can exclude DAG tasks that actually are schedulable but not according to this capacity-bound. & Introduces a new bound called the util-tensity bound
        which proves to be a better schedulability test for GEDF with federated scheduling. & 
        only looks at GEDF with federated scheduling and not other scheduling algorithms. & 
        The scheduling algorithm applied uses GEDF for very low tensity tasks,
        tasks with high utilization and relatively high tensities are scheduled using federated scheduling,
        and low utilization tasks with relatively high tensities are scheduled by partitioned EDF\\
        \hline
        \cite{JiangDecompoSchedParallelTask}  & Decomposition-based scheduling can improve schedulability
        for DAG task scheduling but can also worsen it.
        It is, along with global scheduling, one of the main
        method to schedule DAG tasks. & Develop a decomposition strategy
        as well as a metric / schedulability test 
        and the decomposition strategy proves to be the most efficient one
        according to the defined metric. The scheduling algorithm
        derived from this decomposition strategy (using GEDF variants) shows promising
        results in terms of acceptance ratios & Only looks at GEDF variants which is based on the EDF heuristics for priority assignments.& The decomposition works by first defining execution segments
        and then assigning subtasks to those segments based on their laxity so that there are no segments overloaded with workload. \\
        \hline
        \cite{He2023DegreeOfParallelism} & The notion of degree of parallelism has been used 
        for DAG task scheduling but lacks a clear definition in the research community. & Proposes a new response-time bound for DAG tasks
        as well as a new scheduling algorithm based on federated scheduling that outperforms the SOTA
        by more than 18\% on average & They don't say which intra-task scheduling algorithm is used (just that it's work-conserving)
        and they don't consider intra-task scheduling. & Using the defined notion of degree of parallelism,
        they modify the federated scheduling approach by having a better way of choosing
        the number of cores for heavy tasks, which is based on the degree of parallelism of the heavy DAG task. \\
        \hline
        \cite{Shi2024DagExecGroups}  & Although several DAG intra-task scheduling algorithm have been proposed in the literature,
        most of them ignore the communication delays, specifically the inter-core communications
        between the subtasks of a DAG tasks. In the Robotic / automotive industry,
        most communications can be done using the L1 cache of a processor, hence having a way to group
        subtasks into execution groups to execute on a single processor to remove inter-core communication delays. 
        & Extend the DAG task model to EG-DAG (execution group dag) which binds groups of subtasks to a single physical core,
        thus reducing inter-core communication delays. They also introduce a scheduling algorithm
        and a wcrt for the algorithm while comparing the makespan to existing methods such as federated scheduling and 
        critical-path based sub-tasks scheduling. Their method shows comparable performance
        while minimizing the communication overheads.
         & The evaluation has been done using only 100 DAG tasks which 
         is quite low to cover all different types of DAG tasks.
         They propose a way to schedule multiple DAGs but do not offer
         evaluation results for that. & They use list scheduling with one list per execution group 
         and use worst-fit heuristic to map the execution groups to the processors. \\
        \hline
        \cite{Guan2023FederatedNew}  & Federated scheduling has shown great potential
        for constrained deadline tasks but for arbitrary deadline DAG tasks,
        especially those where the WCET is longer than the period,
        processor assignment is tricky and existing work have shown limitations
        by letting jobs migrate between the assigned processors,
        which produces a more pessimistic schedulability analysis. & 
        Propose a new federated scheduling algorithm for arbitrary deadline DAG tasks
        with WCETs longer than their periods.
        Evaluate the proposed algorithm and comparing it to other scheduling
        algorithms, effectively outperforming most of them in terms of acceptance ratio 
        & Doesn't tackle the problem of resource wasting when using federated scheduling
        or their new version of it. & The new federated algorithm is used when the deadlines are bigger than the periods,
        and the tasks have high densities (according to the classic federated scheduling algorithm).
        For  the high densities tasks that have a deadline lower than their period,
        they use standard federated scheduling and for the low density tasks,
        they use EDF-FF. \\
        \hline
        \cite{Zhao2024GATDRLmodel} & The NP-hard aspect of DAG multi-core scheduling
        makes the optimal solutions (using ILP) time-consuming.
        Hence the researchers have looked at heuristic 
        to have scalable solutions. & Uses Deep reinforcement learning 
        to construct a model that attempts to learn the optimal (in terms of makespan) scheduling policy
        for DAG tasks and compares it to the mathematically optimal ILP method. & Doesn't compare the  machine learning 
        method with SOTA heuristics, but only compares with ILP. & They use a combination of Graph neural network
        with attention layers to better capture the structure and dependency information.
        They use the negative value of the makespan as the reward function to be maximized.\\
        \hline
        \cite{Xu2023DRLtaskSched} & Current methods
        to allocate shared resources on multi-core real-time systems
        use either static analysis of tasks or heuristics
        which cannot represent all possible system usage scenarios,
        thus potentially producing higher WCETs and worse system schedulability. & 
        They use Deep Reinforcement learning to propose a holistic scheduling and allocation
        framework and their model shows better schedulability than 
        existing methods. & Only considers independent periodic tasks
        and also only considers even-EDF and even-RM when comparing schedulability performance. &
        The plaform model is a LLC architecture with a shared memory bus 
        and the DRL model uses a dense network with the proximal policy optimization
        algorithm for training. The DRL model produces a time-triggered 
        schedule table for each tasks' execution and each tasks' memory allocation.\\
        \hline
        \cite{Zhao2022DAGsched} & A previous paper \cite{zhao2020DAGsched} introduced a fixed-priority scheduling 
        algorithm for DAG intra-task scheduling which performed better 
        than SOTA but didn't extend the method to multi-DAG scheduling. & 
        Extends the Concurrent Producer and Consumer (CPC) model from \cite{zhao2020DAGsched} to multi-DAG task scheduling
        by proposing a new multi-DAG scheduling algorithm 
        based on a Parallelism-aware workload distribution model
        which outperforms existing methods in terms of system schedulability. & 
        Only considers constrained deadlines DAG tasks. & Uses a critical path first execution model by assigning 
        the highest priorities to the node in the critical path
        and treating them as providers of parallel execution time
        for the 'consumers' which basically are the parrallelizable subtasks
        for each section of the critical path.
        For multi-DAG scheduling, they use 
        a method similar to federated scheduling but 
        instead of heavy and light tasks,
        they look at what they call the degree of parallelism of the DAG tasks.  \\
        \hline
        \cite{Lee2021GlobalDagSchedDRL} & Several heuristics for DAG intra-task scheduling
        have been used but no scalable optimal scheduling algorithm exists. &
        Propose a Deep Reinforcement Learning based machine learning model 
        that computes an intra-task priority list
        for single DAG task scheduling which outperforms SOTA by up to 3\%
        in terms of makespan & No comparison is made with ILP methods
        that lead to the mathematically minimum makespan.
        Also, it doesn't show the scalability 
        of the GoSu DRL model when increasing the amount of 
        cores in the system. & The network 
        is comprised of a Graph Convolutional network to encode 
        the graph information as well as a sequential decoder
        based on the attention-mechanism to produce a priority list.
        The model is trained using REINFORCE with the negative makespan as a reward function.  \\
        \hline
        \cite{Jiang2023SchedVirtualProcs} & 
        Virtual scheduling, using threads as virtual processors, 
        has been considered in the past but never for DAG task scheduling.
        The similar federaded scheduling method 
        suffers from resource wasting. & Use the concept of virtual processors
        to provide a virtually-federated scheduling algorithm
        which significantly outperforms other federated scheduling methods
        in terms of acceptance ratio. & Only considers implicit and constrained deadline DAG tasks
        and doesn't consider the running time overhead that the proposed method induces. 
        & Introduce the concept of an active Virtual processor 
        and a passive virtual processor (VP) and assign one of each on each physical core.
        The active VP execute the high priority (equivalent of high density in federated scheduling) tasks and the unused processing
        time of the active VP is treated as a passive VP on which low priority
         (equivalent of low density in fedrated scheduling) or high priority task can execute.\\
        \hline
        \cite{GuanFluidDag2022} & Most studies on DAG use the implicity deadline but few are looking
        at DAGs with arbitrary deadline, especially in the case where the deadline 
        is greater than the task's period.
        Fluid scheduling has showed promising results but has only
        been applied to DAG tasks with implicit deadlines. & Propose a fluid scheduling based algorithm
        for constrained and arbitrary deadlines DAG tasks. 
        Introduce the first capacity bound for DAG with deadlines greater than their periods.
        Their algorithm performs better in terms of acceptance ratio
        than SOTA (at the time of 2022). & As for every fluid scheduling based algorithm
        the issue of runtime overhead is not 
        entirely considered as they don't evaluate this metric. & They first decompose each DAG task
        into segments of sequential tasks and then assign execution rates
        to each tasks or threads. Those two steps aim 
        at producing a fluid schedule so that it appears as though each DAG task
        is continuously running on the cores. \\
        \hline
        \cite{GuanFRTDS2020RL} & Most scheduling algorithm that consider
        resource use consider it as constraints rather than considering them 
        as part of the scheduling decision process. & Propose a DRL-based algorithm
        for task scheduling on real-time simulation system called FRTDS.
        The proposed algorithm performs better than existing algorithm on the FRTDS platform
        in terms of makespan for single DAG task scheduling. & The reinforcement learning 
        algorithm uses the previous tasks' execution as experience
        which implies a lot of memory usage,
        thus affecting the speed of execution. & It uses I/O usage and ram allocation
        to construct a cost function which is used as a reward for the RL process.
        The current cost is combined with a future cost, predicted using the successors of
        the current subtasks. \\
        \hline
        \cite{JiangVirtuallyFederatedSched2021} & Federated scheduling 
        has shown great potential for scheduling DAG tasks but 
        suffers from a resource wasting problem which has 
        been addressed but to a limited extent. & Propose a virtually-federated scheduling algorithm
        that efficiently tackles the resource wasting problem of federated scheduling
        while having the same advantages as federated scheduling.
        The algorithm performs better than existing algorithms in terms of acceptance ratio. 
        & Only considers heavy tasks and constrained deadline DAG tasks. 
        & Introduce two virtual processor per physical core, the Active VP and Passive VP
        that are complementary. The active VP has the priority in terms of 
        processing capacity and for the processing capacity 
        not used by an active VP is given to the corresponding passive VP.
        Next, the active VP are allocated to tasks using the difference between
        their deadline and the critical path's length
        as well as the minimal number of processor to schedule a task in federated scheduling.
        Then the passive VPs are allocated to tasks
        according to how useful they are for scheduling the specific task.\\
        \hline
        \cite{Pazzaglia2021DMALETtransfer} & The LET paradigm
        is great at dealing with memory contentions 
        and I/O determinism.
        But typical implementations of LET 
        require local buffers for each core which write/copy data to/from
        global memory. This can be costly when dealing 
        with huge amount of data, like sensor data in autonomous driving system.
        Direct Memory Access (DMA) is a possible solution to such a problem
        but LET hasn't yet been considered with the DMA protocol. &
        Propose a DMA-based protocol to handle LET communications
        on multicore systems that minimizes the read/write latency of each tasks.
        Also, they provide a scheduling algorithm for scheduling the communications
        using MILP to provide an optimal schedule
        which improved by up to 98\% the communication delays compared
        to the classic Giotto approach of LET. & Doesn't consider 
        the scalability of their method in terms of the number of tasks/data transfers to be scheduled. 
        Also, they doesn't provide
        evaluation of known or existing DAG scheduling algorithm
        using their LET communication protocol compared to the same algorithm not using it.
        & For the protocol, for each core, an LET task is responsible for 
        programming the DMA engine so that LET communications can happen.
        For the scheduling algorithm and data allocation,
        a Mixed-Ingeteger Linear Programming (MILP) problem 
        is solved to minimize either the number of DMA data communications,
        or the maximum communication delay to period ratio of each tasks.  \\
        \hline
    \caption{SLR summary table}
    \label{tab:slt_sum_table}
\end{longtable}


\printbibliography

\end{document}

