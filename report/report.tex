\documentclass[conference]{IEEEtran}
\IEEEoverridecommandlockouts
% The preceding line is only needed to identify funding in the first footnote. If that is unneeded, please comment it out.
\usepackage{cite}
\usepackage{amsmath,amssymb,amsfonts}
\usepackage{algorithmic}
\usepackage{graphicx}
\usepackage{textcomp}
\usepackage{xcolor}
\usepackage{longtable}
\usepackage{stmaryrd}
\usepackage[numbers]{natbib}

\pagestyle{plain}
\newcommand{\paraheading}[1]{{\color{blue}\textit{#1}}}
\usepackage{hyperref}

\def\BibTeX{{\rm B\kern-.05em{\sc i\kern-.025em b}\kern-.08em
    T\kern-.1667em\lower.7ex\hbox{E}\kern-.125emX}}
\begin{document}

\title{Scheduling DAG tasks on multicore real-time systems: a supervised learning approach\\
%{\footnotesize \textsuperscript{*}Note: Sub-titles are not captured in Xplore and
%should not be used}
%\thanks{Identify applicable funding agency here. If none, delete this.}
}

\author{\IEEEauthorblockN{Félicien Fiscus-Gay (ID: 23194149)}
\IEEEauthorblockA{\textit{Computer Science} \\
\textit{Auckland University of Technology}\\
Auckland, New Zealand \\
felicien.fgay@gmail.com}
}

\maketitle
\thispagestyle{plain}

\begin{abstract}
Tasks on real-time systems are becoming more and more complex and
those systems are expanding their processing resources and becoming multicore, complexifying
the scheduling of such time-triggered tasks.
To tackle this growth, the Directed Acyclic Graph (DAG) task model has been introduced and
research has been done on designing heuristics
to address the scheduling of DAGs on a multicore platform.
However, very few papers used machine learning for this problem.\\
In this research, the real-time DAG task scheduling literature is reviewed 
and a supervised machine learning model is designed
using a graph convolutional network, and evaluated 
for scheduling a single DAG task with the goal of minimizing its makespan 
using non-preemptive, global fixed-priority scheduling.
The Integer Linear Programming method has been 
used to compute the output labels for the supervised model to learn on
and a non-preemptive, global fixed-priority scheduler simulator 
has been implemented to compute the exact makespan of a DAG task.
The results showed that the supervised model doesn't learn
and that the supervised learning method seems unfit 
for this kind of problem.
The learning issue is mainly due to the computation of 
the output labels, which will be
investigated further in future works.
\end{abstract}

\begin{IEEEkeywords}
real-time system, scheduling, time-triggered tasks, DAG, multicore, machine learning, supervised learning
\end{IEEEkeywords}


\section{Introduction}
\label{sec:intro}

Real-time systems are utilized in various domains such as air traffic 
control, public transportation, and automated vehicles. Unlike non-real-time systems, 
tasks in real-time systems must be both functionally correct and meet strict 
(or flexible) execution time constraints, known as deadlines. Failure 
to meet these deadlines can lead to severe consequences. The critical 
nature of these systems necessitates designing the system 
architecture with a focus on time and incorporating fault tolerance 
to ensure high reliability.

One example of such architecture is the time-triggered 
architecture (TTA)\cite{kopetz2003tta}\cite{kopetz1998timetriggered}, which offers a fault-tolerant communication protocol and a precise timing system to synchronize different electronic control units. Developing and running tasks on these architectures require in-depth knowledge of the system and its architecture, which complicates code reusability and scalability when adding hardware resources or upgrading to a larger system.

To address these issues, the Automotive Open System Architecture 
(AUTOSAR\footnote{\url{https://www.autosar.org/}}) was developed. 
AUTOSAR introduces layers of abstraction between hardware, firmware, 
and software, enhancing software reusability and hardware 
scalability across different systems while maintaining safety and 
security standards. It is now the most widely used architecture among car 
manufacturers, with notable core partners including BMW, Ford, and Toyota.

Scalability, in particular, plays a crucial role in modern 
real-time systems. Increasingly, real-time systems such as 
autonomous cars or computer vision systems are enhancing their 
computational resources by transitioning to multiprocessor systems. 
This shift from uniprocessor to multiprocessor systems addresses 
the growing complexity and computational demands of tasks executed 
on these systems, aiming to reduce both the execution time of these 
tasks and the required resources\cite{maiza2019survey}.

Hence, an increasing number of real-time systems are 
utilizing multi-core hardware to parallelize their tasks 
and convert sequential programs into parallelized ones using 
frameworks such as OpenMP
\footnote{OpenMP (2011) OpenMP Application Program Interface v3.1. 
\url{http://www.openmp.org/mp-documents/OpenMP3.1.pdf}}. 
Unfortunately, in most real-life scenarios, the number of available 
processors/cores is fewer than the number of tasks/subtasks that 
can be executed in parallel (i.e., independent tasks). This means 
that not all independent tasks can be executed simultaneously on 
the system, raising the question: which task should be executed first?

This question is particularly important in a real-time context 
because having the wrong execution order, or schedule, could lead 
to, at best, a slow system, and at worst, deadline misses, which 
can have fatal repercussions. In the case of a self-driving car 
system, for instance, a slight delay of 500 ms in detecting a pedestrian 
crossing the road can, in some cases, be enough to drive over 
the pedestrian or cause a car accident. Note that the resources of 
real-time systems are scarce and limited, which is why using as 
little processing power as possible while ensuring that tasks meet 
their deadlines is of crucial importance.

The extreme case of this scheduling problem arises when only one 
processor is available to execute tasks. This is known as task 
scheduling on a uniprocessor, and \cite{liu1973scheduling} 
provided two major priority policies: Rate Monotonic (RM) and 
Earliest Deadline First (EDF) for scheduling periodic tasks. 
However, when considering multiple processors, the scheduling 
problem becomes much more complex, and different task models must 
be considered.

A prevalent task model is the time-triggered task model, 
which specifies tasks that execute periodically and is well-suited 
for time-triggered systems. Another type of task model is the 
Directed Acyclic Graph (DAG) task model which arises when a time-triggered task
can be parallelized into subtasks which are the nodes of the graph.
Those nodes have dependency constraints which are modeled by the directed edges between the nodes.
The DAG task model is used to model tasks that are parallelizable\cite{baruah1993PFair},
fitting the ever-increasing multicore architectures found in today's real-time systems. 

Given that the problem of scheduling independent tasks or dependency-constrained groups of jobs (i.e., DAGs) is NP-hard\footnote{If a problem is 
NP-hard, it means that it is very unlikely to find a solution in 
polynomial time complexity, i.e., solutions are not scalable}\cite{du1989schedNPhard}
\cite{ULLMAN1975NPhard}, 
people have resorted to either heuristics 
to partially solve the problem,
or the optimal but not scalable Integer Linear Programming
(ILP) method.

Consequently, machine learning will be considered here as it can 
better approximate the unattainable perfect solution while being 
scalable in terms of computing time after the training phase. 
The research questions are:

\begin{itemize}
    \item [RQ1] What is the current state-of-the-Art for DAG tasks scheduling ?        
    \item [RQ2] What machine learning  techniques are used for DAG task scheduling ?
    \item [RQ3]  Can machine learning be a better solution to schedule DAG tasks ?
            \begin{itemize}
                \item [RQ3.1] Can a machine learning solution compare to state-of-the art heuristics for scheduling Directed Acyclic Graph tasks ?
                \item [RQ3.2] Can a machine learning solution compare to an ILP solution while being more scalable ?
            \end{itemize}    
\end{itemize}

To achieve this, the background section will introduce various 
technical terms, concepts, and fundamental algorithms. 
Following this, a systematic literature review will be conducted to address RQ1 and RQ2, 
and finally, the artifact and experimental design, results, and conclusion will 
be presented to answer RQ3.



\paraheading{The solution we propose has the following features..}

\paraheading{The primary contributions of this paper are:}


\section{Background}
\label{sec:bg}

DAG task scheduling introduces several fundamental concepts.

%\subsection{Periodic task and schedule}
%~
%
%Firstly, a periodic task $\tau_i(C_i, D_i, T_i)$ is characterized 
%by its worst-case execution time (wcet) $C_i$, its deadline $D_i$, and 
%its period $T_i$. This definition can be expanded by including an 
%initial offset, which corresponds to the time of the task's first 
%execution, and an activation offset, which is the time delay between 
%the task being ready to execute (i.e., its execution period has begun) 
%and the task actually starting to run. Secondly, a schedule $S$ is a 
%function that assigns a boolean value for each task $\tau$ and each 
%time tick $t$, indicating whether the task $\tau$ is running at 
%time $t$. Therefore, a scheduling algorithm is the method that, 
%given a set of tasks, produces a schedule $S$ for the task set.
%
%This task model and schedule definition are widely adopted in the literature (see section \ref{sec:literature}) 
%and are the building blocks of all scheduling algorithms.
%The periodic task model, in particular, is used to define more complex
%tasks such as DAG tasks (see below) that will be used as input in the 
%machine learning model (see section \ref{sec:methodology}).

\subsection{DAG task model}
~

A Directed Acyclic Graph (DAG) task
$\tau = (G, D, T)$ is comprised of 
a directed acyclic graph $G = (V, E)$, a relative deadline $D > 0$ and a period $T > 0$.
The graph $G$ has a set of vertices $V$ and a set of edges $E \subseteq V \times V$
that models the precedence constraints of the vertices,
i.e., for all $(v_i, v_j) \in V^2$, with $i \neq j$, $(v_i, v_j) \in E$ if and only if
$v_j$ must wait the end of $v_i$'s execution to start executing(i.e., $v_i$ is a precedence constraint for $v_j$).
The nodes of the graph $G$ are characterized by their own worst-case execution time $C_{v_i}$ 
and $vol(G) = \sum_{v_i \in V} C_{v_i}$ is the worst-case execution time of the DAG task $\tau$,
also called the total workload of the DAG task.
A \textit{source} node is the node which only has successors 
and no predecessors, and a \textit{sink} node is 
the node that only has predecessors and no sucessors.
It is assumed that all DAGs have a unique source node and a unique sink node.
This can always be assumed as you can always add a dummy sink or source node,
setting their wcet to 0,
at the end or beginning of a DAG that has multiple sink or source nodes,
linking those non-unique sink and source nodes to
the dummy nodes added so that the dummy nodes become the unique source node and sink node
of the DAG.
In this paper, the words nodes, vertices and subtasks will be used
to address the same thing, that is the vertices of the graph of a DAG task.\\

An example of a DAG task is shown in Figure \ref{fig:dag_example},
where the DAG task has a wcet of 24 time units.

\begin{figure}[htbp]
    \centering
    \includegraphics[width=0.5\linewidth]{images/example_DAG.png}
    \caption{DAG task $\tau$. The worst-case execution time (wcet) of each subtask is written as an exponent
    and the nodes highlighted in red are the nodes in the critical path, the path of maximum length in terms of wcets.}
    \label{fig:dag_example}
    \includegraphics[width=\linewidth]{images/example_DAG_schedules.png}
    \caption{Execution schedules for the DAG task above. Heavily inspired from \citet{zhao2020DAGsched}.}
    \label{fig:dag_schedule_example}
\end{figure}


The DAG task model will be the task model used in order to conduct 
the systematic literature review (see Section \ref{sec:literature})
and it will also be the task model used for designing 
the supervised machine learning model (see Section \ref{sec:methodology}).

\subsection{Makespan}

The makespan or end-to-end response time of a 
DAG task is the amount of time it takes for all the subtasks
in the DAG task to finish executing when given a schedule.
For instance, for the task $\tau$ shown in Figure \ref{fig:dag_example},
the makespan of $\tau$ for the first schedule (from top to bottom) shown in Figure \ref{fig:dag_schedule_example}
is 17.
Notice that Figure \ref{fig:dag_schedule_example} shows multiple
ways of scheduling the DAG task, the bottom one yielding a makespan of 13.

This is a key measurement when scheduling the nodes of a single DAG task,
also called intra-task scheduling (see Section \ref{sec:literature}),
and it will be the main efficacy criteria when comparing 
the machine learning model with state-of-the-Art heuristics and ILP
(see Section \ref{sec:methodology}).

\subsection{Intra-task and inter-task scheduling}
~

When considering the DAG task model,
two concepts arise: intra-task scheduling and inter-task scheduling.
Intra-task scheduling of DAG tasks is when only the execution
order of the nodes in a single DAG task is considered.
The goal of such a problem being to minimize the makespan
of a single DAG task.

Inter-task scheduling, on the other hand, is when multiple DAG tasks
are considered and the goal then becomes to maximize the acceptance ratio (see below)
of the scheduling algorithm.

\subsection{NP-hardness intuition and problem motivation}
~

The problem of intra-task scheduling of DAG tasks,
that is, scheduling the nodes of a single DAG task so that 
the makespan is minimized,
is a NP-hard problem\cite{ULLMAN1975NPhard}\cite{du1989schedNPhard}
and can also be intuitively seen as such when looking at 
the schedules shown in Figure \ref{fig:dag_schedule_example}.
Indeed, 
when looking at a single DAG task,
a simple yet efficient way of scheduling the nodes 
is looking at a critical-path-first execution order(CPFE).
The critical path, highlighted in red in Figure \ref{fig:dag_example},
is the longest path of the graph in terms of wcets.
The CPFE principle is about assigning the highest execution priority
to the nodes that are part of the critical path in the graph.
Basically, if multiple nodes, including a node in the critical path, are ready for execution,
the node in the critical path will be executed first.
The CPFE principle is applied in the third and fourth schedules (from top to bottom)
in Figure \ref{fig:dag_schedule_example} and achieves a makespan of 14 in the third schedule.
However, this is not the minimum makespan as the fourth schedule yields a makespan
of 13.

This is due to the fact that, although executing the critical path first
is the best strategy\cite{zhao2020DAGsched}, 
one needs to take into account the dependency constraints
and thus the global structure of the graph.
The fourth schedule in Figure \ref{fig:dag_schedule_example}
achieves a makespan of 13 by prioritizing the execution of node $v_8$
before node $v_5$ which is crucial to enable $v_3$ to execute sooner
than in the third schedule.
Typically, in graph theory, when the global structure needs to be known 
in advance
in order to find the best solution, it often means that the problem is NP-hard
(e.g., the problem of coloring a graph\cite{book1975richardNPhardColorGraph}
or coloring a path of a graph\cite{ERLEBACH2001ColorPathNPhard}).
Therefore, researchers have resorted to heuristics to compute an execution order
that minimizes the bound of the DAG, reducing its wcet 
by providing a better worst-case makespan bound than $vol(G)$ (see Section \ref{sec:literature}).


\subsection{Global and partitioned scheduling}
~

In task scheduling,
the term \textit{global} means that 
we allow the tasks to migrate between multiple processors or cores
during their execution, i.e., the tasks are
not bound to a single processor.
On the other hand, the term \textit{partitioned}
in scheduling means that we don't allow tasks to migrate 
and rather fix each of them on a specific processor to execute on
indefinitely.

\subsection{Priority scheduling}
~

Multiple scheduling algorithm use either fixed-priority 
on non-fixed priority scheduling.
Firstly, priority scheduling is the concept 
of assigning different execution priorities 
to different tasks and, when choosing which tasks 
must execute first, executing the one with the highest priority.
Fixed-priority scheduling means that the priority values
are assigned to each task before the scheduling starts
and are not changed afterwards(e.g., the RM algorithm\cite{liu1973scheduling}).
On the other hand, non-fixed or online-priority scheduling
is when the tasks' priorities are computed during 
the scheduling process(e.g., the EDF algorithm\cite{liu1973scheduling}).

\subsection{Epochs and batch size}
~

In machine learning,
an \textit{epoch} is a training phase iteration where 
the model goes through the entire training dataset, 
the model, then, went through one epoch.
The training phase often consists of multiple epochs to 
enhance the training performance of the model.
A \textit{batch} is a subset, or chunk,  of the training dataset
that is given to the model to learn on and adjust its weights.
Hence, the \textit{batch size} is the number of data samples
the model should look at before it learns from its mistakes
and adjusts itself.
When using stochastic gradient descent, for instance,
the model first computes the average gradient of the batch
it just processed and then uses this mean value to 
tweak its weights.

\subsection{Utilization factor}

The utilization factor represents the percentage of processing 
time that a taskset $(\tau_1, \cdots, \tau_n)$ will utilize. 
Formally, it is defined as
\begin{align}
U = \sum_{k=1}^{n} \frac{vol(G_k)}{T_k}
\end{align}
where $U$ is the utilization factor. This concept is significant 
because, when evaluating a scheduling algorithm $S$, we desire 
$S$ to effectively schedule tasksets that maximize the utilization 
factor $U$. Consequently, the higher the utilization factor bound 
for $S$, the more efficient the scheduling algorithm. Additionally, 
this concept is valuable in real-time systems where processing 
resources are often limited and expensive, making it crucial to 
maximize their usage.

This concept is also used either as a measurement
when comparing two scheduling algorithms 
and considering their minimum-utilization bound,
or used as a parameter to generate tasksets or DAG tasks with 
a fixed utilization (see Section \ref{sec:literature}).


\subsection{Capacity augmentation bound}

Another measurement used when scheduling DAG tasks  
is the capacity bound, or capacity augmentation bound,
which compares the use of computing resources to a theoretically optimal scheduling algorithm.
It can also be used as a simple schedulability test.
It is often refer to as a $\beta$ coefficient and the lower $\beta$ is, the better the scheduling algorithm.\\
This metric is used in the literature when scheduling DAG tasks (Section \ref{sec:literature}).

%\cite{tiryaki2006sunit} is really good with MAS but really bad with something else.
%\subsection{Optimality}
%
%A scheduling algorithm $S$ is said to be optimal 
%when the following condition is true:
%for every taskset $\Omega$, if there exists 
%a scheduling algorithm $S'$ so that $\Omega$ is feasible by $S'$,
%then $\Omega$ is also feasible by $S$.
%Where {\it{feasible}}, 
%means that, using the schedule generated by $S$,
%all the tasks in the taskset will finish executing before their deadlines.
%
%This concept is used in the literature, mainly for independent tasks scheduling
%(see Section \ref{sec:literature}).

\subsection{Acceptance ratio}

When dealing with several independent DAG tasks
or tasksets, 
the acceptance ratio is often used to measure the 
performance of a scheduling algorithm (see Section \ref{sec:literature}).
It consists of looking at a number of generated tasksets (or DAG tasks)
and calculating the amount of schedulable (i.e., 
the schedule produced doesn't lead to a deadline miss) tasksets compared to 
the total amount of taskets.
The resulting percentage is the acceptance ratio 
and the closer it gets to 100\% for a given scheduling algorithm, the better the scheduling algorithm.

This concept is thus used as a metric, to assert the efficiency
of scheduling algorithms when considering inter-task scheduling (see Section \ref{sec:literature}).
\\


While the acceptance ratio, also called system schedulability, is used 
to measure the performance of scheduling algorithms for inter-DAG task scheduling
(i.e., scheduling multiple DAG tasks),
the makespan and the capacity bound are only used for single DAG tasks.

%\subsection{RM and EDF scheduling}
%~
%
%When designing a scheduling algorithm, the key decision involves 
%determining which task should execute first when two or more 
%independent tasks are ready to execute. This requires assigning each 
%task a priority. \cite{liu1973scheduling} introduced two 
%heuristics for this purpose: Rate Monotonic (RM) and Earliest 
%Deadline First (EDF).
%
%The RM algorithm is a fixed-priority scheduling algorithm, 
%meaning that the priority of each task is known before execution 
%begins. RM assigns the highest priority to tasks with the minimum 
%execution rate, i.e., $\frac{C_k}{T_k}$, and is considered optimal 
%for assigning fixed priorities to tasks. In contrast, EDF assigns 
%priorities dynamically by selecting tasks based on which one has 
%the earliest absolute deadline.
%
%Figure \ref{fig:edf_rm_examples} illustrates the difference 
%between the two algorithms by scheduling the same two tasks, 
%$\tau_1$ and $\tau_2$. $\tau_1$ has a worst-case execution time of 0.5 time units 
%and a period of 2 time units, while $\tau_2$ has a worst-case execution 
%time of 2 time units and a period of 3 time units. These are 
%examples of implicit deadline tasks, where the relative deadline 
%equals the end of their execution period. 
%
%\begin{figure}
%    \centering
%    \includegraphics[width=\linewidth, height=100px]{images/schedule_rm.png}
%    a)
%    \includegraphics[width=\linewidth, height=100px]{images/schedule_edf.png}
%    b)
%    \caption{Schedules of $\tau_1$ and $\tau_2$ using Rate Monotonic (a)
%    and Earliest Deadline First (b) heuristics.}
%    \label{fig:edf_rm_examples}
%\end{figure}
%
%Although EDF calculates each priority at runtime, it is optimal 
%for uniprocessor scheduling and has a theoretical utilization bound 
%of 1, which is the maximum possible for a feasible taskset on a 
%single processor. RM, on the other hand, has a much lower 
%utilization bound than EDF. While one might argue that RM introduces 
%less runtime overhead and is therefore more practical, it has been 
%shown that RM leads to more task preemptions (interrupting the 
%execution of a task, as seen at times 2 and 4 for task $\tau_2$ in 
%Figure \ref{fig:edf_rm_examples}.a). This, combined with its lower 
%utilization bound and non-optimality, makes EDF perform better than 
%RM\cite{buttazzo2005RMvsEDF}.
%
%Although \cite{liu1973scheduling}'s work focused on uniprocessor 
%systems, the proposed algorithms have also been applied to 
%multi-processor scheduling.
%For example, Global EDF (GEDF)
%can be used on multi-core systems when allowing task migrations
%and Partitioned EDF (PEDF) is used when forbidding task migrations
%(The RM equivalents also exist).

\section{Related Works}
\label{sec:literature}

\subsection{Systematic Literature Review process}

\paraheading{\textbf{Scoping}}

This SLR aims at tackling RQ1. More precisely, the following research questions will be answered:

\begin{itemize}
    \item [RQ1.1] What is the current state-of-the-Art for DAG task scheduling with precedence constraints ?
    \item [RQ1.2] How has LET been used in scheduling event-chains ?
    \item [RQ1.3] What machine learning techniques have been used for scheduling tasks on real-time systems ?
\end{itemize}
It will also be shown how the literature doesn't provide 
a complete answer to RQ2, hence the contributions of this paper.\\

From these research questions, several concepts have been isolated,
namely, time-triggered tasks, the nature of the system (real-time multicore system),
the scheduling of tasks, DAG tasks, and machine learning.
The recording of the search results were done using the BibTeX LateX plugin
combine with the google scholar "cite" feature.

Searching was conducted using the IEEE and ACM databases.
According to the concepts identified above, 
the keyword chain used for searching was 
"("real-time" OR "real time") AND 
"system" AND ("time-triggered" OR "time triggered" OR "DAG" OR "Directed Acyclic Graph" OR "LET" OR "Logical Execution Time" OR "event chain" OR "event-chain") 
AND "task" 
AND ("scheduling" OR "scheduler" OR "schedule") 
AND ("multi-processors" OR "multi-cores" OR "multi processors" OR 
"multi cores" OR "multi-processor" OR "multi processor" OR 
"multi-core" OR "multi core")".

The search produced 3,549 results on the IEEE database

exclusion : past 5 years : IEEE --> down to 1,171 
            heterogeneous not in title + abstract : IEEE --> down to 999
            mixed critical* not in title + abstract : IEEE --> down to 952
            scheduling or scheduler or schedule in title but not "energy" :  IEEE --> down to 155 and 149 when just considering conference and journal papers (not early access)
            
            removing those not about real-time system,
            not about proposing a scheduling algorithm,
            not about DAG nor LET tasks or event-chains : IEEE --> down to 21
            After reading the complete articles --> 19.

\paragraph{}

\subsection{Findings of the Literature Review}

\paraheading{The works reviewed were compared on the following metrics.} 
\begin{itemize}
    \item \textbf{Utilization Bound}: useful to see which algorithm is more efficient at using the available resources.
    \item \textbf{Acceptance Ratio}: it shows how optimal (see Section \ref{sec:bg}) a scheduling algorithm is.
    \item \textbf{Makespan}: for DAG task scheduling, widely used in the literature.
    \item \textbf{Runtime Overhead}: some scheduling algorithms can show promising results theoretically but are practically very slow because 
    of their complexity adding runtime overhead on the scheduler, this metric will not be a number but rather an amount such as minimal, practical, non-practical.
\end{itemize}
Every metric used here have also been chosen for their prevalence in the literature.

A comparison of the works was carried out and the overall results are illustrated in Table~\ref{tab:slrtable}.
 
In \cite{guan2021DAGfluid}, authors 
use fluid scheduling to schedule multiple DAG tasks
on a multicore system. 
Fluid scheduling has been used in previous work for independent
time-triggered task scheduling\cite{baruah1993PFair}\cite{cho2006LLREF} 
but very few consider DAG tasks. Fluid-scheduling is known for producing optimal scheduling algorithms.
Their method decomposes a DAG task into several sequential segments
in which the subtasks will execute according to the fluid scheduling 
model. Although their algorithm significantly outperforms
existing algorithms, the main limitation that is common to all fluid-based
scheduling algorithm is the runtime overhead induced by the fluid-scheduling model.
Although authors in \cite{guan2021DAGfluid} briefly explain how 
to transform their scheduling algorithm to a non-fluid one for practical implementation,
they do not evaluate the overhead caused by the frequent task migrations
and preemptions. Also, their algorithm only considers DAG task with implicit deadlines
(D = T) which makes the response-time analysis simpler but to the cost
of generalizability.

As a follow up, authors in \cite{GuanFluidDag2022} 
extend the fluid scheduling algorithm in \cite{guan2021DAGfluid}
to constrained and arbitrary deadline tasks,
especially focusing on DAG tasks with a deadline greater than their period.
Their main contributions are their new scheduling algorithm that 
performs better than existing methods in terms of acceptance ratio,
and producing the first theoretical capacity bound for DAG tasks
with deadlines greater than their periods.
However, the authors still don't provide any evaluation 
on the amount of runtime overhead their scheduling algorithm implementation
produces which generally lowers the actual acceptance ratio
of the algorithm. 

Instead of considering fluid-scheduling,
a popular scheduling method is federated scheduling.
Federated scheduling is based on the idea
of assigning heavy tasks ($U > 1$) to multiple cores
for the whole duration of the tasks' executions,
and assigning light tasks ($U \le 1$) to execute on
cores that have not been assigned a heavy task.
Although it is popular, it suffers from a resource wasting problem,
especially when the difference between the critical path's length 
and the deadline is small,
which many papers aim at solving
\cite{Guan2023FederatedNew}
\cite{jiangUtilTensityBound}
\cite{JiangVirtuallyFederatedSched2021}
\cite{Jiang2023SchedVirtualProcs}
\cite{Kobayashi2023FedBundledDagsched}
\cite{Shi2024DagExecGroups}
\cite{He2023DegreeOfParallelism}.

\cite{jiangUtilTensityBound}, for instance, 
consider federated scheduling and GEDF
and introduces a better metric called the util-tensity bound
that extends the concept of capacity bound
to have a better schedulability test.
Based on this newly derived bound, the authors 
propose an extension to the classic federated algorithm,
with very low tensity tasks being scheduled with GEDF, 
tasks with high-utilization and relatively high tensities are scheduled
using the classic federated scheduling and low utilization
tasks with relatively high tensities are scheduled using partitioned-EDF.
Their algorithm, based on their newly derived bound, effectively improves
the system schedulability of DAG tasks and reduces the resource wasting 
problem of federated scheduling. The main limitation
of this paper is that they only consider GEDF 
for their util-tensity bound and also only consider implicit deadline DAG tasks.

This problem of resource wasting in federated scheduling
is also tackled in \cite{Kobayashi2023FedBundledDagsched}
where the authors propose a federated and bundled-based scheduling
algorithm which enhances the schedulability of DAG tasks
compared to existing federated scheduling algorithms.
Their method consists of using federated scheduling for
tasks with high critical path to deadline ratio and bundled
scheduling for tasks with low critical path to deadline ratio.
Unfortunately, this paper only looks at 3 DAG tasks to evaluate
their algorithm which is a really small amount and is not 
representative of the different DAG tasks that can exist.

Authors in \cite{JiangVirtuallyFederatedSched2021}
take another approach by proposingg a virtually-federated 
scheduling algorithm that leverages the advantages
of federated scheduling while improving the acceptance 
ratio for DAG tasks, outperforming existing algorithms.


\begin{table}
    \centering
    \begin{tabular}[]{|l|p{0.20\linewidth}|p{0.20\linewidth}|p{0.20\linewidth}|}
        \hline
        \textbf{Reference} & \textbf{Scheduling technique} & \textbf{Task type} & \textbf{Scope (intra/inter/both)}\\
        \hline
        \cite{guan2021DAGfluid} & fluid & implicit deadline & inter\\
        \hline
        \cite{He2019DagIntra} & priority-list & constrained deadline & intra \\
        \hline
        \cite{Kobayashi2023FedBundledDagsched} & federated and bundled-based & constrained deadline & inter\\
        \hline
        \cite{Xiao2019}  & clustering & constrained deadline & intra\\ 
        \hline
        \cite{Igarashi2020HeuristicContentionFree}  & priority-list & LET constrained deadline & both \\
        \hline
        \cite{jiangUtilTensityBound}  & federated and GEDF and PEDF & implicit deadline & inter\\
        \hline
        \cite{JiangDecompoSchedParallelTask} & Decomposition-based & implicit deadline & inter \\
        \hline
        \cite{He2023DegreeOfParallelism} & federated-based & constrained deadlines & inter \\
        \hline
        \cite{Shi2024DagExecGroups}  & partitioned / clustering & constrained deadlines & intra\\
        \hline
        \cite{Guan2023FederatedNew}  & federated & arbitrary deadline & inter\\
        \hline
        \cite{Zhao2024GATDRLmodel} & DRL & constrained deadline & intra\\
        \hline
        \cite{Xu2023DRLtaskSched} & DRL & non-DAG implicit deadline & inter\\
        \hline
        \cite{Zhao2022DAGsched} & priority-list and federated & constrained deadline & both \\
        \hline
        \cite{Lee2021GlobalDagSchedDRL} & DRL & constrained deadline & intra\\
        \hline
        \cite{Jiang2023SchedVirtualProcs} & federated-based & constrained deadline & inter\\
        \hline
        \cite{GuanFluidDag2022} & fluid & constrained/arbitrary deadline & inter\\
        \hline
        \cite{GuanFRTDS2020RL} & DRL & constrained deadline & intra \\
        \hline
        \cite{JiangVirtuallyFederatedSched2021} & federated-based & constrained deadline & inter\\
        \hline
        \cite{Pazzaglia2021DMALETtransfer} & Mixed ILP & LET, constrained deadline & inter\\
        \hline
        \textbf{Total: 19} & \textbf{DRL: 4, Federated: 7, Fluid: 2, ILP: 1, Priority-List(intra): 3, Clustering: 2, Decomposition: 1}
        & \textbf{implicit: 4, constrained: 13, arbitrary: 2} & \textbf{inter: 11, intra: 6, both: 2} \\
        \hline
    \end{tabular}
    \caption{SLR summary table}
    \label{tab:slt_sum_table}
\end{table}





\section{Research Methodology}
\label{sec:methodology}

To answer the research questions defined in section \ref{sec:intro},
the right research methodology needs to be selected to conduct 
systematic research.
Multiple such methodologies exist but the following four will be presented
and the ones which fit best to this research will be selected.

\subsection{Systematic Literature Review}

The systematic literature review (SLR) methodology aims at
looking at the current state-of-the-Art in a specific domain by selecting a range 
of articles in the scientific literature\cite{KITCHENHAM2009SLR}.
The main difference between a simple literature review and 
a SLR is that the results need to be reproducable,
that is, one can get the exact same results by simply following the different steps described by the one writing the SLR.
This is done by first defining a set of keywords to search for, i.e., a search string, 
and then identifying the databases to search on.
The resulting articles are then filtered out using exclusion and/or inclusion criterias
to narrow down the number of papers to review.
Examples of such criterias are restricting the publication year-range,
excluding certain types of articles (i.e., conferences, early-access, etc.), etc.
The remaining articles are then screened by first reading 
their abstract and then, from the resulting filtered articles, their entire content,
which further filters the articles found through the initial search.
This process gives a final list of papers to review and compare against each other 
to correctly answer the research questions. 
It is especially useful to answer research questions such as RQ1 and RQ2
and to have a good representation of the state-of-the-Art
but it will often show the research gaps that exists in the literature,
thus not fully answering a specific research question.
To address these gaps, other research methods need to be considered.

\subsection{Design Science}

The design science methodology is focused on the creation and evaluation 
of artifacts intended to solve practical problems. It emphasizes 
the development of knowledge through the design and implementation 
of innovative solutions, such as algorithms, methods, tools, 
and frameworks, particularly in fields like software engineering\cite{Wieringa2010DesignScienceMethod}.
For instance, one might design a new machine learning architecture or a new algorithm to 
better answer a specific problem, while rigorously detailing the design and evaluation aspect of this new technology or system.
Indeed, in this methodology,
the designed artifact(s) need(s) to answer a precise problem,
defined beforehand, and choices in the design must be clearly justified.
The evaluation of the artifact(s) must not only answer research questions, but also
demonstrate practical applications,
thus balancing methodological rigor and practical utility.
In our case, the problem of scheduling DAG tasks
has important practical applications and according to the SLR, 
no one looked at supervised learning as a solution to generating priority-lists for 
global fixed-priority scheduling.
The non-existence of a supervised learning model implies it must be created, or designed,
if we want to evaluate such a model,
which is exactly the purpose of the design science methodology.
Hence, RQ3 will be answered by designing a supervised learning model and evaluating this potential solution to the fixed-priority DAG task scheduling problem.

\subsection{Case study}

A case study is useful at providing a great understanding of a real-life scenario that happened 
or that is currently happening, illustrating a specific issue with a real-life context.
In a case study, you look closely at real-life observations to evaluate or investigate
a design in the context of a real-life scenario\cite{sarah2011caseStudy}.
It is especially used in the medical field to study and investigate a group of people and their 
eating habits or look at the effect of placebo in a control group compared to a medical subtsance to be tested
on a test group.
One can also use a case study methodology to study a specific phenomena and how an application react to it
which can lead, in the case of a software application, to finding
limitations or missing elements.
Another use of the case study methodology is 
to address one or more real-life issues by analyzing 
a system using multiple datasets.
For instance, a case study might design a benchmark for testing 
certain types of applications.
In this research, no real-life scenario will be considered as 
the data collected will be via a DAG task generator because real-life data collection is very time-consuming
and isn't the focus of this study.
Indeed, the focus will be on designing an artifact,
rather than investigating an existing one, simply because there aren't any supervised 
learning model that has been found in the literature (see Section \ref{sec:literature}). 
Therefore, the case study approach will not be used in this research.

\subsection{Experiment}
 
 
The experiment, or controlled experiment, methodology aims at identifying cause-effect 
chains by defining hypotheses and control variables on a specific system,
and running multiple tests or experiments while having fixed and varying variables to assert 
their role in the system's behaviour\cite{basili2007controlledExperiment}.
The controlled experiment approach 
can provide insights on the relationships between the controlled variables and the system
which contributes to the body of knowledge of the specific system.
This can be used, for instance, to observe and explain the impact of hyperparameters of a machine learning
model on its performance.
This approach has its drawbacks, the main one being 
that the experiment always has biases and doesn't fully reflect the reality
of the matter. This drawback can be exacerbated when considering 
a small sample size or forgetting
to take into account controlled variables that impact the results
of the experiment, which is why great care needs to be taken 
for the design phase of the experiment(s).
The controlled experiment approach could be used in this research to validate or invalidate hypotheses
about the impact of hyperparameters in the AI model's performance,
experimenting with different values and hyperparameters for the model and the dataset generation.
However, the time-consuming aspect of such experiments
combined with the fact that 
the focus of this study is more on the design and evaluation of the model rather than 
experimenting with its hyperparameters,
is the reason why the controlled experiment methodology will not be used in this paper.
\\

The SLR can answer research questions RQ1 and RQ2,
by first collecting and comparing the different papers in the literature
and then identifying not only the state-of-the-Art for scheduling DAG tasks on real-time systems,
but also the machine learning techniques that are currently used in the literature as well as their drawbacks and limitations.
The fact that only very few papers use machine learning to tackle the problem of real-time DAG task scheduling
motivates the design of a new machine learning model to better answer RQ3 which implies 
applying the design science methodology to provide a more complete answer as to how a supervised machine learning model 
can compare to heuristics and also the ILP method.
The time limitations as well as the focus of this research being on the evaluation of the model
justifies the controlled experiment methodology not being necessary to answer RQ3, and thus not considered here. 
Furthermore, the time-consuming aspect of the case-study approach as well as the fact that the identified SOTA papers
use a dag task generation script and not real-life dag tasks datasets, makes
the use of the case study methology not relevant in this research's context and not able to correctly compare the results with 
the SOTA papers.

\section{Problem definition}
~

The problem to be tackled is the problem of scheduling a single DAG task with a non-preemptive, global and fixed-priority scheduler (GFPS),
on a multicore system. That is, the problem of computing an execution priority for each node of a DAG task so that it minimizes, 
using a non-preemptive GFPS, the makespan of the DAG.

More precisely, given a DAG task $\tau = G = (V, E)$ and $n = \left\lvert V \right\rvert$, the designed machine learning model 
ideally needs to compute a list of priorities $p^* \in \llbracket 0, n-1 \rrbracket^{n}$
so that 
$$
    \text{AFT}(v_{\text{sink}}, p^*) = \min_{p \in \llbracket 0, n-1 \rrbracket^{n}} \text{AFT}(v_{\text{sink}}, p)
$$

with $\text{AFT}(v_{\text{sink}}, p)$ being the actual finish time of the
sink node of $\tau$, i.e., the makespan of the DAG task $\tau$,
calculated using a non-preemptive GFPS. Note that $p^*$ is not necessarily unique.
In this paper, the priorities will be higher when having a lower value,
that is, the highest priority given to a node is 0 and the lowest is $n-1$.

%\subsection{Hyper Parameters}
%~
%
%\begin{list}{}{}
%    \item - \citet{Zhao2024GATDRLmodel} : procs = 4, num nodes = [10, 20, 30, 40, 50], num dag tasks = 1000 for training and 600 for testing (1600 total)
%    \item - \citet{Lee2021GlobalDagSchedDRL} experiment : procs = [2, 3, 4, 6, 8], num dag tasks = 8000 for training | 1000 validation | 1000 testing (10000 total)
%    \item -  \citet{zhao2020DAGsched} dag gen: layer min = 5, layer max = 8, parallelism = Unif(2, [4,5,6,7,8]) (with procs=4), connect prob = 0.5,
%                num dag tasks = 1000  but doesn't say what's the total workload they're using. 
%\end{list}
\section{System design}
\label{sec:system_design}
~

No supervised learning model has been designed to 
tackle the non-preemptive, global fixed-priority scheduling problem 
for DAG tasks.
Therefore, we must first design a supervised learning model.

In supervised learning, the machine learning model 
predicts an output (forward pass) and then computes the error between
the predicted output and a known true output.
The gradient of the error function is then propagated through the network to tweak the model's 
trainable parameters, which is also called the backward pass (see Figure \ref{fig:supervised_learning}).

\begin{figure}
    \centering
    \includegraphics[width=\linewidth]{images/supervised_learning_diagram.drawio.png}
    \caption{The supervised learning method.}
    \label{fig:supervised_learning}
\end{figure}

\subsection{Supervised design}
~

\subsubsection{Input DAGs}
~

Each input DAG  task will be represented using 
a matrix of numbers with each row being a node
and each column being a raw feature of a node.
The list of raw features is similar to what is proposed by \citet{Lee2021GlobalDagSchedDRL},
that is :
\begin{list}{}{}
    \item - the normalized wcet of the node, i.e., $C_i / L$ for node $i$ with $L$ being the total workload ;
    \item - the number of incoming neighbours ;
    \item - the number of outgoing neighbours ;
    \item - a boolean value of whether the node is the source or sink node ;
    \item - a boolean value of whether the node is part of the critical path of the DAG.
\end{list}

The wcet is widely used as a criteria for priority-list scheduling algorithm (RM, EDF\cite{buttazzo2005RMvsEDF}, etc.)
but it needs to be normalized to have the context information of the whole graph.
The number of incoming and outgoing neighbours makes it possible 
to take into account the inner structure of the graph to compute the execution order.
The source and sink nodes are particular nodes in that their priority
is not important because they will respectively execute first and last due to their dependency constraints.
Finally, the critical path plays a huge role in state-of-the-art heuristics\cite{He2019DagIntra}\cite{zhao2020DAGsched},
with nodes in the critical path often having higher priorities than those
that aren't.
An example of such a DAG task representation is shown in Figure \ref{fig:dag_task_matrix_example}.

\begin{figure}
    \centering
    \includegraphics[width=\linewidth]{images/dag_matrix_example.drawio.png}
    \caption{Example dag task with 5 nodes a total workload of 12
    and the matrix representation of each node with the column respectively being the above list of features.}
    \label{fig:dag_task_matrix_example}
\end{figure}

\subsubsection{Output labels}
~

An Integer Linear Programming (ILP) solver will be used to compute
the optimal (minimum makespan) schedule for each DAG task and then
ordering the nodes according to their release time in the ILP schedule.

The output of the model will be a matrix of probabilities,
with each row being the index of a node and each column being the index of the priority.
There are as many priorities as there are nodes and 
the priority list of each DAG is then retrieved using the column
index of the maximum probability as the assigned probability, for each row.
Matrix output example for the DAG task shown in Figure \ref{fig:dag_task_matrix_example}
is shown in Figure \ref{fig:dag_output_matrix_example}.

\begin{figure}
    \centering
    \includegraphics[width=\linewidth]{images/output_matrix_example.drawio.png}
    \caption{Example of an matrix output from the model, the predicted list of priorities (on the right)
    is retrieved from the probabilities.}
    \label{fig:dag_output_matrix_example}
\end{figure}

The ILP output matrix to compare the predicted output to is of the same shape
as the predicted output matrix with the probabilities being 1 on the optimal priority for each row 
, and 0 otherwise.

\subsubsection{Loss function}
~

The problem is being treated as a classification problem,
hence the binary cross-entropy loss function will be used.
This function is defined as follows :
\begin{equation}
    loss(x, y) = \sum_{i=1}^{n} -y_i\log(x_i)
\end{equation}
    
where $x$ is the flattened matrix representing the predicted output,
$y$ is the flattened matrix representing the true output (ILP output),
and $n$ is the number of element in $x$ and $y$.


\subsection{Model design}
~

The model's architecture is very similar to the proposed encoder in \citet{Lee2021GlobalDagSchedDRL},
that is, the model is comprised of three modules.
Two feed forward networks and one attention-based graph convolutional network (see Figure \ref{fig:model_diagram}).

\begin{figure}
    \centering
    \includegraphics[width=\linewidth]{images/designed_model.png}
    \caption{The architecture diagram of the proposed supervised machine learning model.}
    \label{fig:model_diagram}
\end{figure}

\subsubsection{Feed forward networks}
~

The two feed forward networks have 3 layers with each 
layer $n$ producing the following output,
where $W_n \in \mathbb{R}^{5\times5}$ is the matrix of trainable weights
at layer $n$, $b_n \in \mathbb{R}^5$ is the bias and $X_n$ is either a node's vector representation
or the output of the previous layer, and $ReLU(x) = max(0, x)$:
\begin{equation}
    o_{n} = ReLU(W_{n}X_{n} + b_n),\, n \in \{1,2,3\}
\end{equation}
There is an exception for the second feed forward network where the last 
layer's output is computed using the following equation:
\begin{equation}
    o_{3} = \tanh(W_{3}X_{3} + b_3)
\end{equation}
where $W_3 \in \mathbb{R}^{nbPriorities \times 5}$,
$b_3 \in \mathbb{R}^{nbPriorities}$ and $\tanh$ is the 
hyperbolic tangent activation function.
Also, $nbPriorities$ corresponds to the number of different priorities
that can be assigned to each node.


\subsubsection{Graph Convolutional Network}
~


For each gcn layer, the input vector $X_k$ 
goes through an aggregation phase where,
given the set of incoming neighbours of node $v_i$, $\mathcal{N}_{in}(v_i)$, which includes
$v_i$, and the set of outgoing neighbours $\mathcal{N}_{out}(v_i)$, which also includes $v_i$,
the next vector representation $X_{k+1}$ is calculated by the following equations:

$$
\begin{array}{l}
AttentionModule(X_i) = \sum_{j \in \mathcal{N}(v_i)} \alpha_{ij} X_j
\end{array}
$$

\begin{figure}
    \centering
    \includegraphics[width=0.5\linewidth]{images/gcn_update_aggregate_diagram.png}
    \caption{Diagram of the graph convolution network layer. $X_k$ is the transformed node vector representation
    and ELU is the exponential linear unit function (see Equation \ref{eq:elu}).}
    \label{fig:update_aggregate_diagram}
\end{figure}

\begin{equation}
\text{ELU}(x) = 
\begin{cases}
x, & \text{if } x > 0 \\
\exp(x) - 1, & \text{if } x \leq 0
\end{cases}
\label{eq:elu}
\end{equation}
\section{Evaluation}
~

\subsection{Environment setting}
~

The evaluation of the model is focused on two metrics : the makespan and the computing time once trained.
For the implementation of the model, python 3.10.12 with the PyTorch 2.4.1 library is used and runs on a google collab runtime
with 90 TPU cores.
The stochastic gradient descent is used for training with a learning rate of $0.001$ and a batch size of 250.

During the initial evaluation, an oversmoothing problem occured which lead to reducing the number of GCN layers to 1(see Section \ref{sec:model_design}).
Unfortunately reducing the number of GCN layers wasn't enough which lead 
to adding a regularization term to the loss function.
The idea is to tackle the oversmoothing probblem by forcing the model
to differentiate between the nodes' representation.
Hence, the regularization term is the squared inverse of the average euclidian distance 
between each of the nodes' latent representation, i.e., 
\begin{equation}
    \text{regTerm}(X) = \frac{1}{\sum_{i}\sum_{j > i} \Vert X_i - X_j\Vert_{2} }
\end{equation}
    
and the loss function then becomes
\begin{equation}
    loss(X, y) = \sum_{i} -y_i\log(x_i) + \text{regTerm}(X)
\end{equation}
with $x_i$ being the elements in $x$, the flattened matrix representation of $X$
and $y$ the true output (see Section \ref{sec:loss_design}).

\subsection{Dataset generation}
~

To generate the DAG tasks, the generator from \citet{zhao2020DAGsched} is used, which is also the one used in \citet{Lee2021GlobalDagSchedDRL}.
The random DAGs are generated using the following process :
The generator starts at a source node and expands outward, 
creating nodes in successive layers. The total number of layers, 
or maximum depth, is randomly determined to be between two values $a$ and $b$.
For each layer, the number of nodes generated is chosen uniformly, 
ranging from 2 up to the parallelism parameter, $p$ which in this case, 
is fixed at $p=8$. Nodes that do 
not already have connections can randomly connect to other nodes in 
the previous layer with a probability of $p_c=0.5$. After all layers 
are generated, any terminal nodes are linked to a final sink node. 
Both the source and sink nodes are used to structure the graph and 
have a fixed execution time of one unit each. Lastly, 
execution times are assigned randomly to all nodes while ensuring 
that the total workload sums up to $W = 1000$\cite{zhao2020DAGsched}.

To generate DAGs with a fixed number of nodes $n$, 
the generator is first used to generate 50000 DAG tasks
with different values for $a$ and $b$ depdending on what 
the value of $n$ is. Then, the generated DAGs with 
the specified number of nodes are retrieved from the dataset.
Specifically, Table \ref{tab:layer_num_minmax} 
shows the different $a$ and $b$ values according to $n$.

\begin{table}
    \centering
    \begin{tabular}{|c|c|c|}    
        \hline
        \textbf{n} & \textbf{a} & \textbf{b} \\
        \hline
        10 & 3 & 8 \\
        \hline
        \{20, 30\} & 5 & 8 \\
        \hline
        40 & 7 & 10 \\
        \hline
        50 & 10 & 15 \\
        \hline
    \end{tabular}
    \caption{dag generator $a$, minimum number of layers, and $b$, maximum
    number of layers, parameter values for generating 
    random DAGs according to number of fixed nodes per graph we need to retrieve afterwards.}
    \label{tab:layer_num_minmax}
\end{table}

Using those values, 1400 DAG tasks were retrieved and used for 
evaluation, for each value of $n$.
1000 of them were used for training the model, 400 for testing
and 100 of them were used to measure the computing time 
for both the ILP and the supervised ML methods.

\section{Limitations}
~
\\


Although the DAG generator used is indeed used in the literature\cite{Lee2021GlobalDagSchedDRL}\cite{zhao2020DAGsched}\cite{Zhao2022DAGsched},
the generator is not made for generating a number of DAG tasks
with a fixed number of nodes. Hence, 
using the generator as it has been used might not
lead to a dataset as diverse as what is done by \citet{Zhao2024GATDRLmodel} for instance.
Also, no evaluation of the model using multiple values of the parallelism
parameter $p$ has been done, which could've been interesting 
to look at to see the impact of this parameter, and of how parallelizable 
a DAG task is, on the performance.
The amount of DAG tasks used as a test set is 400 which 
is lower than what the other papers used, 600 in \citet{Zhao2024GATDRLmodel}
and 1000 in \citet{Lee2021GlobalDagSchedDRL}, which makes randomness
have a bigger influence than in those papers.

For the evaluation, 
it has only been done using a train set and test set,
no cross-validation such as K-fold has been done which could have
enhanced the interpretation of the results, reducing the impact
of dataset bias in the performance results.

Furthermore, 
the supervised learning method might still be an option
for this kind of problem, as the main problem isn't the supervised
learning method per se, but rather the computation of the output labels. 
Also, although the ILP method computes
optimal schedules, it does take a long time to do so and is not scalable,
making the model training not scalable which can if we want to train the model
on DAGs with 100, 500 or 1000 nodes.



\section{Conclusions and Future Works}


In this paper, the DAG task scheduling on multicore systems problem
has been addressed, especially looking at the machine learning
techniques used to tackle this problem.
It has been found that very few papers use machine learning
for this kind of problem and no paper has been found using 
supervised learning for the single DAG multicore scheduling problem.
Hence, a supervised learning model has been designed and evaluated.
Unfortunately, the performance results 
showed how the supervised learning method seems unfit for this kind of 
problem, with the main issue being the computation of the output labels.
Reinforcement learning, on the other hand, appears to yield much better results, comparable
to state-of-the-Art heuristics and the optimal Integer Linear Programming method.

Future works should be done on computing unique and more informative output labels,
to expand on the results' discussion,
and, if successful, expand the model to tackle multi-DAGs scheduling as well. 
One could also work on using reinforcement learning for 
multi-rate DAGs, or chain of time-triggered tasks, and potentially
extend the task model to include other types of tasks,
such as Logical Execution Time tasks.










\section*{Acknowledgement}

A special thank you to my supervisor Dr. Matthew Kuo
who guided me throughout this dissertation,
as well as my friend Alexi Cannesse who helped 
me to debug some code when implementing the machine learning model.


\bibliographystyle{IEEEtranN}
\bibliography{references}


\end{document}
