
\section{Research Methodology}
\label{sec:methodology}

To answer the research questions defined in section \ref{sec:intro},
the right research methodology needs to be selected to conduct 
systematic research.
Multiple such methodologies exists but the following four will be evaluated
and the ones which fit best to this research will be selected.

\subsection{Systematic Literature Review}

The systematic literature review (SLR) methodology aims at
looking at the current state-of-the art in a specific domain by selecting a range 
of articles in the scientific literature\cite{KITCHENHAM2009SLR}.
The main difference between a simple literature review is that the results need to be reproducable,
that is, one can get the exact same results by simply following the different steps described by the one writing the SLR.
This is done by first defining a set of keywords to search for, i.e., a search string, 
and then identifying the databases to search on.
The resulting articles are then filtered out using exclusion and/or inclusion criteria
to narrow down the number of papers to review.
Examples of such criterias are restricting the publication year-range,
excluding certain types of articles (i.e., conferences, early-access, etc.), etc.
The remaining articles are then screened by first reading 
their abstract and then, from the resulting filtered articles, their entire content,
which further filters the articles found through the initial search.
This process gives a final list of papers to review and compare against each other 
to correctly answer the research questions. 
It is especially useful to answer research questions such as RQ1 and RQ2
and to have a good representation of the state-of-the-Art
but often will show the research gaps that exists in the literature,
thus not fully answering a specific research question.
To address these gaps, other research methods need to be considered.

\subsection{Design Science}

The design science methodology is focused on the creation and evaluation 
of artifacts intended to solve practical problems. It emphasizes 
the development of knowledge through the design and implementation 
of innovative solutions, such as algorithms, methods, tools, 
and frameworks, particularly in fields like software engineering\cite{Wieringa2010DesignScienceMethod}.
For instance, one might design a new machine learning architecture or a new algorithm to 
better answer a specific problem, while rigorously detailing the design and evaluation aspect of this new technology or system.
Indeed, in this methodology,
the designed artifact(s) need(s) to answer a precise problem,
defined beforehand, and choices in the design must be clearly justified.
The evaluation of the artifact(s) needs not only to answer research questions, but also
demonstrate practical applications,
thus balancing methodological rigor and practical utility.
In our case, the problem of scheduling DAG tasks
has important practical applications and according to the SLR, 
no one looked at supervised learning as a solution to generating priority-lists for 
global fixed-priority scheduling.
The non-existence of a supervised learning model implies it needs to be created, or designed,
which is exactly the purpose of the design science methodology.
Hence, RQ3 will be answered by designing a supervised learning model and evaluating this potential solution to the fixed-priority DAG task scheduling problem.

\subsection{Case study}

A case study is useful at providing a great understanding of a real-life scenario that happened 
or that is currently happening, illustrating a specific issue with a real-life context.
In a case study, you look closely at real-life observations to evaluate or investigate
a design in the context of a real-life scenario\cite{sarah2011caseStudy}.
It is especially used in the medical field to study and investigate a group of people and their 
eating habits or look at the effect of placebo in a control group compared to a medical subtsance to be tested
on a test group.
One can also use a case study methodology to study a specific phenomena and how an application react to it
which can lead, in the case of a software application, to finding
limitations or missing elements.
Another use of the case study methodology is 
to address one or more real-life issues by analyzing 
a system using multiple datasets.
For instance, a case study might design a benchmark for testing 
certain types of applications.
In this research, no real-life scenario will be considered as 
the data collected will be via a DAG task generator because real-life data collection is very time-consuming
and isn't the focus of this study.
Indeed, the focus will be on designing an artifact,
rather than investigating an existing one, simply because there aren't any supervised 
learning model that has been found in the literature (see Section \ref{sec:literature}). 
Therefore, the case study approach will not be used in this research.

\subsection{Experiment}
 
 
The experiment, or controlled experiment, methodology aims at identifying cause-effect 
chains by defining hypotheses and control variables on a specific system,
and running multiple tests or experiments while having fixed and varying variables to assert 
their role in the system's behaviour\cite{basili2007controlledExperiment}.
The controlled experiment approach 
can provide insights on relationships between the controlled variables and the system
which contributes to the body of knowledge of the specific system.
This can be used, for instance, to observe and explain the impact of hyperparameters of a machine learning
model on its performance.
This approach has its drawbacks, the main one being 
that the experiment always has biases and doesn't fully reflect the reality
of the matter. This drawback can be exacerbated when considering 
a small sample size or forgetting
to take into account controlled variables that impact the results
of the experiment, which is why great care needs to be taken 
for the design phase of the experiment(s).
Nonetheless, the controlled experiment approach will be used in this research to validate or invalidate hypotheses
about the designed AI model performances, especially when compared to the ILP method.
It will answer RQ3 by not only evaluating the model designed using the design science methodology, 
but also experimenting with different values and parameters that are not part of the evaluation,
to better grasp the behaviour and performances of the designed model.
The drawback mentioned previously will be tackled using a random dag generator as well as a 
large DAG task set, to keep the observational bias to a minimum.
Cross-validation of the model will also be used to minimize the impact of this drawback.\\


The SLR can answer research questions 1 and 2,
by first collecting and comparing the different papers in the literature
and then identifying not only the state-of-the-Art for scheduling DAG tasks on real-time systems,
but also the machine learning techniques that are currently used in the literature as well as their drawbacks and limitations.
The fact that only very few papers use machine learning to tackle the problem of real-time DAG task scheduling
motivates the design of a new machine learning model to better answer RQ3 which implies 
applying the design science methodology to provide a more complete answer as to how a supervised machine learning model 
can compare to reinforcement learning models, heuristics, and also the ILP method.
In addition, the controlled experiment approach can help interpret the evaluation results of the designed model by 
playing with different hyperparameter sets and also  
 providing an answer to the scalability aspect in RQ3.2.
The time limitations of this study as well as the fact that the identified SOTA papers
use a dag task generation script and not real-life dag tasks datasets, makes
the use of the case study methology not relevant in this research's context and not able to correctly compare the results with 
the SOTA papers. 


\section{Problem definition}
~

The problem to be tackled is the problem of scheduling a single DAG task with a non-preemptive, global and fixed-priority scheduler (GFPS),
on a multicore system. That is, the problem of computing a priority for each node for a DAG task so that it minimizes, 
using a non-preemptive GFPS, the makespan of the DAG.

More precisely, given a DAG task $\tau = G = (V, E)$ and $n = \left\lvert V \right\rvert$, the designed machine learning model 
idealy needs to compute a list of priorities $p^* \in \llbracket 0, n-1 \rrbracket^{n}$
so that 
$$
    \text{AFT}(v_{\text{sink}}, p^*) = \min_{p \in \llbracket 0, n-1 \rrbracket^{n}} \text{AFT}(v_{\text{sink}}, p)
$$

with $\text{AFT}(v_{\text{sink}}, p)$ being the actual finish time of the
sink node of $\tau$, i.e., the makespan of the DAG task $\tau$,
calculated using a non-preemptive GFPS. Note that $p^*$ is not necessarily unique.
In this paper, the priorities will be higher when having a lower value,
that is, the highest priority given to a node is 0 and the lowest is $n-1$.

%\subsection{Hyper Parameters}
%~
%
%\begin{list}{}{}
%    \item - \citet{Zhao2024GATDRLmodel} : procs = 4, num nodes = [10, 20, 30, 40, 50], num dag tasks = 1000 for training and 600 for testing (1600 total)
%    \item - \citet{Lee2021GlobalDagSchedDRL} experiment : procs = [2, 3, 4, 6, 8], num dag tasks = 8000 for training | 1000 validation | 1000 testing (10000 total)
%    \item -  \citet{zhao2020DAGsched} dag gen: layer min = 5, layer max = 8, parallelism = Unif(2, [4,5,6,7,8]) (with procs=4), connect prob = 0.5,
%                num dag tasks = 1000  but doesn't say what's the total workload they're using. 
%\end{list}