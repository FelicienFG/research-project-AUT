
\section{Research Methodology}
\label{sec:methodology}

To answer the research questions defined in section \ref{sec:intro},
the right research methodology needs to be selected to conduct 
systematic research.
Multiple such methodologies exists but the following four will be evaluated
and the ones which fit best to this research will be selected.

\subsection{Systematic Literature Review}

The systematic literature review (SLR) methodology aims at
looking at the current state-of-the art in a specific domain by selecting a range 
of articles in the scientific literature\cite{KITCHENHAM2009SLR}.
This is done by first defining a set of keywords to search for, i.e., a search string, 
and then identifying the databases to search on.
The resulting articles are then filtered out using exclusion and/or inclusion criteria
to narrow down the number of papers to review.
Examples of such criterias are restricting the publication year-range,
excluding certain types of articles (i.e., conferences, early-access, etc.), etc.
The remaining articles are then screened by first reading 
their abstract and then, from the resulting filtered articles, their entire content,
which further filters the articles found through the initial search.
This process gives a final list of papers to review and compare against each other 
to correctly answer the research questions. 
It is especially useful to answer research questions such as RQ1 and RQ2
and to have a good representation of the state-of-the-Art
but often will show the research gaps that exists in the literature,
thus not fully answering a specific research question.
To address these gaps, other research methods need to be considered.

\subsection{Design Science}

The design science methodology is focused on the creation and evaluation 
of artifacts intended to solve practical problems. It emphasizes 
the development of knowledge through the design and implementation 
of innovative solutions, such as algorithms, methods, tools, 
and frameworks, particularly in fields like software engineering\cite{Wieringa2010DesignScienceMethod}.
In this methodology,
the designed artifact(s) need(s) to answer a precise problem,
defined beforehand, and choices in the design must be clearly justified.
The evaluation of the artifact(s) needs not only to answer research questions, but also
demonstrate practical applications,
thus balancing methodological rigor and practical utility.
In our case, the problem of scheduling DAG tasks
has important practical applications and there are space 
for evaluating new artifacts, to better answer RQ3.

\subsection{Case study}

A case study is useful at providing a great understanding of a real-life scenario that happened 
or that is currently happening, illustrating a specific issue with a real-life context.
In a case study, you look closely at real-life observations to evaluate or investigate
a design in the context of a real-life scenario\cite{sarah2011caseStudy}.
One can use a case study methodology to study a specific phenomena and how an application react to it
which can lead, in the case of a software application, to finding
limitations or missing elements.
Another use of the case study methodology is 
to address one or more real-life issues by analyzing 
a system using multiple datasets.
For instance, a case study might design a benchmark for testing 
certain types of applications.
In this research, no real-life scenario will be considered and the focus will be on designing an artifact,
rather than investigating an existing one. 
Therefore, the case study approach will not be useful to this research.

\subsection{Experiment}
 
 
The experiment, or controlled experiment, methodology aims at identifying cause-effect 
chains by defining hypotheses and control variables on a specific system,
and running multiple tests or experiments while having fixed and varying variables to assert 
their role in the system's behaviour\cite{basili2007controlledExperiment}.
The controlled experiment approach 
can provide insights on relationships between the controlled variables and the system
which contributes to the body of knowledge of the specific system.
This can be used, for instance, to observe and explain the impact of hyperparameters of a machine learning
model on its performance.
This approach has its drawbacks, the main one being 
that the experiment always has biases and doesn't fully reflect the reality
of the matter. This drawback can be exacerbated when considering 
a small sample size or forgetting
to take into account controlled variables that impact the results
of the experiment, which is why great care needs to be taken 
for the design phase of the experiment(s).
Nonetheless, the controlled experiment approach will be used in this research to validate or invalidate hypotheses
about the designed AI model performances, especially when compared to the ILP method.\\


On one hand, the SLR can answer research questions 1 and 2,
by identifying the state-of-the-art and finding gaps in the research.
Gaps that can be filled, in part, by the design science methodology,
designing an artifact that will address this / these gap(s).
On the other hand, the controlled experiment approach
will be used to analyze the behaviour of the artifact
according to controlled variables, 
aiming to answer RQ3.


\subsection{Hyper Parameters}
~

\begin{list}{}{}
    \item - \citet{Zhao2024GATDRLmodel} : procs = 4, num nodes = [10, 20, 30, 40, 50], num dag tasks = 1000 for training and 600 for testing (1600 total)
    \item - \citet{Lee2021GlobalDagSchedDRL} experiment : procs = [2, 3, 4, 6, 8], num dag tasks = 8000 for training | 1000 validation | 1000 testing (10000 total)
    \item -  \citet{zhao2020DAGsched} dag gen: layer min = 5, layer max = 8, parallelism = Unif(2, [4,5,6,7,8]) (with procs=4), connect prob = 0.5,
                num dag tasks = 1000  but doesn't say what's the total workload they're using. 
\end{list}