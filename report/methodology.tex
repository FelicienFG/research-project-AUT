
\section{Research Methodology}
\label{sec:methodology}

To answer the research questions defined in section \ref{sec:intro},
the right research methodology needs to be selected to conduct 
systematic research.
Multiple such methodologies exists but the following four will be evaluated
and the ones which fit best to this research will be selected.

\subsection{Systematic Literature Review}

The systematic literature review (SLR) methodology aims at
looking at the current state-of-the art in a specific domain by selecting a range 
of articles in the scientific literature\cite{KITCHENHAM2009SLR}.
This is done by first defining a set of keywords to search for, i.e., a search string, 
and then identifying the databases to search on.
The resulting articles are then filtered out using exclusion and/or inclusion criteria
to narrow down the number of papers to review.
Examples of such criterias are restricting the publication year-range,
excluding certain types of articles (i.e., conferences, early-access, etc.), etc.
The remaining articles are then screened by first reading 
their abstract and then, from the resulting filtered articles, their entire content,
which further filters the articles found through the initial search.
This process gives a final list of papers to review and compare against each other 
to correctly answer the research questions. 
It is especially useful to answer research questions such as RQ1 and RQ2
and to have a good representation of the state-of-the-Art
but often will show the research gaps that exists in the literature,
thus not fully answering a specific research question.
To address these gaps, other research methods need to be considered.

\subsection{Design Science}



\subsection{Case study}

\subsection{Experiments}
 
 


